\documentclass[12pt,a4paper]{report}
\usepackage[utf8]{inputenc}
\usepackage[bahasa]{babel}
\usepackage{graphicx}
\usepackage{hyperref}
\usepackage{geometry}
\usepackage{fancyhdr}
\usepackage{longtable}
\usepackage{array}
\usepackage{xcolor}
\usepackage{listings}
\usepackage{tocloft}
\usepackage{titlesec}
\usepackage{amssymb}

% Page setup
\geometry{a4paper, margin=2.5cm}
\setlength{\parindent}{0pt}
\setlength{\parskip}{6pt}
\setlength{\headheight}{14.5pt}

% Header and footer
\pagestyle{fancy}
\fancyhf{}
\fancyhead[L]{\leftmark}
\fancyhead[R]{\thepage}
\fancyfoot[C]{User Manual - Web Company Profile PT. Surya Kencana Gemilang Teknik}
\renewcommand{\headrulewidth}{0.4pt}
\renewcommand{\footrulewidth}{0.4pt}

% Hyperlink setup
\hypersetup{
    colorlinks=true,
    linkcolor=black,
    filecolor=magenta,
    urlcolor=cyan,
    pdftitle={User Manual Web Company Profile},
    pdfpagemode=FullScreen,
}

% Title formatting
\titleformat{\chapter}[display]
{\normalfont\huge\bfseries}{\chaptertitlename\ \thechapter}{20pt}{\Huge}
\titlespacing*{\chapter}{0pt}{-20pt}{20pt}

% Colors
\definecolor{primarycolor}{RGB}{25,89,87}
\definecolor{secondarycolor}{RGB}{196,161,87}

\begin{document}

% =====================
% COVER PAGE
% =====================
\begin{titlepage}
    \centering
    \vspace*{2cm}

    {\LARGE\bfseries BUKU PETUNJUK PENGGUNAAN WEBSITE\\[0.3cm]}
    {\Large\bfseries (USER MANUAL BOOK)}\\[1.5cm]

    {\Huge\bfseries WEB COMPANY PROFILE\\[0.3cm]}
    {\Huge\bfseries PT. SURYA KENCANA}\\[2cm]
    {\Huge\bfseries GEMILANG TEKNIK}\\[2cm]

    % \includegraphics[width=0.3\textwidth]{images/logo-company.png}\\[2cm] % Image not found

    {\large\bfseries Versi 1.0}\\[0.5cm]
    {\large November 2025}\\[3cm]

    \vfill

    {\large PT. Surya Kencana Gemilang Teknik\\
    Yogyakarta, Indonesia}

\end{titlepage}

% =====================
% TABLE OF CONTENTS
% =====================
\tableofcontents
\newpage

% =====================
% CHAPTER 1: PENDAHULUAN
% =====================
\chapter{Pendahuluan}

\section{Tentang PT. Surya Kencana Gemilang Teknik}

PT. Surya Kencana Gemilang Teknik adalah perusahaan yang bergerak di bidang penyediaan mesin-mesin industri berkualitas tinggi. Perusahaan ini telah dipercaya oleh berbagai klien dari berbagai sektor industri di Indonesia.

Website Company Profile PT. Surya Kencana Gemilang Teknik dirancang sebagai platform digital untuk:
\begin{itemize}
    \item Menampilkan profil perusahaan secara profesional
    \item Memperkenalkan produk-produk mesin industri kepada calon klien
    \item Menampilkan portofolio klien yang telah bekerja sama
    \item Menyediakan kanal komunikasi langsung dengan calon pelanggan
    \item Meningkatkan kredibilitas dan trust perusahaan
\end{itemize}

\section{Tujuan Dokumen}

Dokumen ini memberikan panduan lengkap mengenai penggunaan website company profile PT. Surya Kencana Gemilang Teknik, mencakup:
\begin{itemize}
    \item Cara mengakses dan menavigasi website sebagai pengunjung (user)
    \item Panduan lengkap untuk administrator dalam mengelola konten website
    \item Prosedur operasional standar untuk setiap fitur yang tersedia
    \item Troubleshooting dan solusi masalah umum
\end{itemize}

\section{Persyaratan Sistem}

\subsection{Spesifikasi Teknis Website}

\begin{table}[h]
\centering
\begin{tabular}{|>{\bfseries}p{4cm}|p{8cm}|}
\hline
\textbf{Komponen} & \textbf{Teknologi / Keterangan} \\
\hline
Frontend Framework & Vue.js 3 dengan TypeScript \\
\hline
Backend Framework & Laravel 11 (PHP 8.2) \\
\hline
Database & MySQL 8.0 \\
\hline
Styling & Tailwind CSS \\
\hline
Icon Library & Font Awesome 6 \\
\hline
Storage & Cloudflare R2 (CDN) \\
\hline
Server & CapRover / Cloud Server \\
\hline
\end{tabular}
\caption{Spesifikasi Teknis Website}
\end{table}

\subsection{Persyaratan Perangkat Pengguna}

Untuk mengakses website ini, pengguna memerlukan:

\textbf{Hardware Minimum:}
\begin{itemize}
    \item Processor: Intel Core i3 atau setara
    \item RAM: 4 GB
    \item Koneksi Internet: Minimum 2 Mbps
\end{itemize}

\textbf{Software:}
\begin{itemize}
    \item Browser Modern: Google Chrome (versi 100+), Mozilla Firefox (versi 100+), Microsoft Edge, atau Safari
    \item Sistem Operasi: Windows 10/11, macOS, Linux, Android, iOS
    \item JavaScript harus diaktifkan
\end{itemize}

\subsection{Akun dan Hak Akses}

Website ini memiliki dua jenis pengguna:

\begin{enumerate}
    \item \textbf{Pengunjung (User/Guest):}
    \begin{itemize}
        \item Tidak memerlukan login
        \item Dapat melihat semua informasi publik
        \item Dapat menghubungi perusahaan via WhatsApp
    \end{itemize}

    \item \textbf{Administrator:}
    \begin{itemize}
        \item Memerlukan akun dan login
        \item Dapat mengelola seluruh konten website (CRUD)
        \item Terdiri dari Admin dan Super Admin
    \end{itemize}
\end{enumerate}

% =====================
% CHAPTER 2: MEMULAI SISTEM
% =====================
\chapter{Memulai Sistem}

\section{Cara Mengakses Website}

\subsection{Akses untuk Pengunjung (User)}

\begin{enumerate}
    \item Buka web browser (Chrome, Firefox, Edge, atau Safari)
    \item Ketik URL website: \texttt{https://becompro.app.fizualstd.my.id/}
    \item Halaman beranda website akan ditampilkan
    \item Navigasi website dapat dilakukan melalui menu di bagian atas (navbar)
\end{enumerate}

\begin{figure}[h]
    \centering
    \includegraphics[width=0.9\textwidth]{images/homepage-user.png}
    \caption{Tampilan Halaman Beranda Website}
\end{figure}

\subsection{Akses untuk Administrator}

\begin{enumerate}
    \item Buka web browser
    \item Ketik URL admin: \texttt{https://becompro.app.fizualstd.my.id/admin/dashboard}
    \item Anda akan diarahkan ke halaman login jika belum masuk
    \item Masukkan kredensial yang valid
\end{enumerate}

\begin{figure}[h]
    \centering
    \includegraphics[width=0.7\textwidth]{images/admin-login-url.png}
    \caption{URL untuk Akses Admin Dashboard}
\end{figure}

\section{Login dan Logout Administrator}

\subsection{Prosedur Login Admin}

Halaman login hanya diperuntukkan bagi administrator. Pengunjung biasa tidak memerlukan login.

\textbf{Langkah-langkah Login:}

\begin{enumerate}
    \item Akses halaman login melalui URL: \\
    \texttt{https://becompro.app.fizualstd.my.id/login}

    \item Masukkan \textbf{Username} yang terdaftar di database

    \item Masukkan \textbf{Password} yang sesuai

    \item Klik tombol \textbf{"Login"}

    \item Sistem akan memverifikasi kredensial Anda

    \item Jika berhasil, Anda akan diarahkan ke Dashboard Admin

    \item Jika gagal, pesan error akan ditampilkan
\end{enumerate}

\begin{figure}[h]
    \centering
    \includegraphics[width=0.8\textwidth]{images/login-page.png}
    \caption{Halaman Login Administrator}
\end{figure}

\textbf{Catatan Keamanan:}
\begin{itemize}
    \item Jangan bagikan username dan password Anda kepada siapapun
    \item Selalu logout setelah selesai menggunakan sistem
    \item Gunakan password yang kuat (minimal 8 karakter, kombinasi huruf besar, kecil, angka)
\end{itemize}

\subsection{Prosedur Logout Admin}

\begin{enumerate}
    \item Klik nama pengguna atau ikon profil di pojok kanan atas
    \item Pilih menu \textbf{"Logout"}
    \item Anda akan diarahkan kembali ke halaman login
    \item Session Anda akan dihapus dari sistem
\end{enumerate}

\begin{figure}[h]
    \centering
    \includegraphics[width=0.5\textwidth]{images/logout-button.png}
    \caption{Tombol Logout pada Header Admin}
\end{figure}

\section{Tampilan Dashboard Admin}

Setelah berhasil login, administrator akan melihat Dashboard Utama yang menampilkan:

\begin{itemize}
    \item Statistik ringkasan (Total Produk, Total Klien, Total Testimoni, Total Admin)
    \item Grafik atau chart jika ada
    \item Menu navigasi sidebar di sebelah kiri
    \item Header dengan informasi user dan tombol logout
\end{itemize}

\begin{figure}[h]
    \centering
    \includegraphics[width=0.9\textwidth]{images/admin-dashboard.png}
    \caption{Tampilan Dashboard Administrator}
\end{figure}

\textbf{Menu Navigasi Sidebar:}

\begin{itemize}
    \item \textbf{Dashboard} - Halaman utama dengan statistik
    \item \textbf{Manajemen Produk} - Kelola produk mesin industri
    \item \textbf{Manajemen Klien} - Kelola data klien perusahaan
    \item \textbf{Manajemen Testimoni} - Kelola testimoni pelanggan
    \item \textbf{Visi \& Misi} - Edit visi dan misi perusahaan
    \item \textbf{Kontak} - Edit informasi kontak
    \item \textbf{Hero Section} - Kelola gambar banner utama
    \item \textbf{Riwayat Perusahaan} - Kelola sejarah perusahaan
    \item \textbf{Pengaturan Situs} - Konfigurasi label dan teks website
    \item \textbf{Manajemen Admin} - Kelola akun admin (khusus Super Admin)
\end{itemize}

% =====================
% CHAPTER 3: PENGELOLAAN DATA ADMIN
% =====================
\chapter{Dashboard Admin - Pengelolaan Data}

Bagian ini menjelaskan secara detail cara melakukan operasi CRUD (Create, Read, Update, Delete) pada setiap menu data yang tersedia di dashboard administrator.

\section{Manajemen Produk}

Fitur ini digunakan untuk mengelola data produk mesin industri yang dijual oleh perusahaan.

\subsection{Melihat Daftar Produk}

\begin{enumerate}
    \item Klik menu \textbf{"Manajemen Produk"} pada sidebar
    \item Daftar produk akan ditampilkan dalam bentuk grid/card
    \item Setiap card menampilkan:
    \begin{itemize}
        \item Gambar produk
        \item Nama produk
        \item Harga
        \item Deskripsi singkat
        \item Tombol Edit dan Hapus
    \end{itemize}
\end{enumerate}

\begin{figure}[h]
    \centering
    \includegraphics[width=0.9\textwidth]{images/produk-list.png}
    \caption{Tampilan Daftar Produk}
\end{figure}

\subsection{Mencari Produk}

\begin{enumerate}
    \item Gunakan kolom pencarian di bagian atas halaman
    \item Ketik nama produk yang ingin dicari
    \item Hasil pencarian akan ditampilkan secara real-time
\end{enumerate}

\subsection{Menambah Produk Baru}

\begin{enumerate}
    \item Klik tombol \textbf{"Tambah Produk"} di bagian atas halaman
    \item Form input produk akan muncul
    \item Isi data produk:
    \begin{itemize}
        \item \textbf{Nama Produk*} - Wajib diisi
        \item \textbf{Harga*} - Masukkan dalam format angka (Rp)
        \item \textbf{Deskripsi*} - Penjelasan lengkap produk
        \item \textbf{Gambar Utama*} - Upload gambar utama produk (Format: JPG, PNG; Max: 5MB)
        \item \textbf{Gambar Tambahan} - Upload gambar galeri (Opsional, max 5 gambar)
    \end{itemize}
    \item Klik tombol \textbf{"Simpan"}
    \item Produk baru akan muncul di daftar
\end{enumerate}

\begin{figure}[h]
    \centering
    \includegraphics[width=0.7\textwidth]{images/produk-form-add.png}
    \caption{Form Tambah Produk}
\end{figure}

\textbf{Catatan Penting:}
\begin{itemize}
    \item Field bertanda * wajib diisi
    \item Gunakan gambar berkualitas tinggi untuk tampilan profesional
    \item Deskripsi produk sebaiknya informatif dan menarik
    \item Harga akan otomatis diformat dengan pemisah ribuan
\end{itemize}

\subsection{Mengedit Data Produk}

\begin{enumerate}
    \item Klik tombol \textbf{"Edit"} pada card produk yang ingin diubah
    \item Form edit akan muncul dengan data yang sudah terisi
    \item Ubah data yang diperlukan
    \item Untuk mengganti gambar:
    \begin{itemize}
        \item Upload gambar baru (gambar lama akan otomatis terhapus)
        \item Atau biarkan kosong jika tidak ingin mengganti
    \end{itemize}
    \item Klik \textbf{"Simpan Perubahan"}
    \item Data produk akan diperbarui
\end{enumerate}

% \begin{figure}[h]
%     \centering
%     \includegraphics[width=0.7\textwidth]{images/produk-form-edit.png}
%     \caption{Form Edit Produk}
% \end{figure} % Image not found

\subsection{Menghapus Produk}

\begin{enumerate}
    \item Klik tombol \textbf{"Hapus"} pada card produk
    \item Dialog konfirmasi akan muncul
    \item Baca peringatan dengan teliti
    \item Klik \textbf{"Ya, Hapus"} untuk konfirmasi
    \item Atau klik \textbf{"Batal"} untuk membatalkan
    \item Produk dan semua gambar terkait akan dihapus permanen
\end{enumerate}

% \begin{figure}[h]
%     \centering
%     \includegraphics[width=0.6\textwidth]{images/produk-delete-confirm.png}
%     \caption{Konfirmasi Hapus Produk}
% \end{figure} % Image not found

\textbf{Peringatan:}
\begin{itemize}
    \item Penghapusan bersifat permanen dan tidak dapat dibatalkan
    \item Pastikan produk yang dihapus memang sudah tidak diperlukan
\end{itemize}

\section{Manajemen Klien}

Fitur ini digunakan untuk mengelola data klien/mitra yang telah bekerja sama dengan perusahaan.

\subsection{Melihat Daftar Klien}

\begin{enumerate}
    \item Klik menu \textbf{"Manajemen Klien"} pada sidebar
    \item Daftar klien akan ditampilkan dalam bentuk grid
    \item Setiap card menampilkan:
    \begin{itemize}
        \item Logo klien
        \item Nama klien
        \item Nama institusi (jika ada)
        \item Tombol Edit dan Hapus
    \end{itemize}
\end{enumerate}

\begin{figure}[h]
    \centering
    \includegraphics[width=0.9\textwidth]{images/client-list.png}
    \caption{Tampilan Daftar Klien}
\end{figure}

\subsection{Menambah Klien Baru}

\begin{enumerate}
    \item Klik tombol \textbf{"Tambah Client"}
    \item Isi form dengan data:
    \begin{itemize}
        \item \textbf{Nama Client*} - Nama perusahaan/organisasi
        \item \textbf{Institusi} - Detail institusi (opsional)
        \item \textbf{Logo*} - Upload logo klien (Format: PNG, JPG; Max: 2MB)
    \end{itemize}
    \item Klik \textbf{"Simpan"}
\end{enumerate}

% \begin{figure}[h]
%     \centering
%     \includegraphics[width=0.7\textwidth]{images/client-form-add.png}
%     \caption{Form Tambah Klien}
% \end{figure} % Image not found

\textbf{Tips Upload Logo:}
\begin{itemize}
    \item Gunakan logo dengan latar belakang transparan (PNG)
    \item Ukuran disarankan: 300x200 pixels
    \item Logo akan otomatis di-resize jika terlalu besar
\end{itemize}

\subsection{Mengedit Data Klien}

\begin{enumerate}
    \item Klik tombol \textbf{"Edit"} pada card klien
    \item Ubah data yang diperlukan
    \item Upload logo baru jika ingin mengganti
    \item Klik \textbf{"Simpan Perubahan"}
\end{enumerate}

\subsection{Menghapus Klien}

\begin{enumerate}
    \item Klik tombol \textbf{"Hapus"} pada card klien
    \item Konfirmasi penghapusan pada dialog yang muncul
    \item Klik \textbf{"Ya, Hapus"}
\end{enumerate}

\section{Manajemen Testimoni}

Fitur untuk mengelola testimoni dari pelanggan yang puas dengan produk/layanan perusahaan.

\subsection{Melihat Daftar Testimoni}

\begin{enumerate}
    \item Klik menu \textbf{"Manajemen Testimoni"} pada sidebar
    \item Testimoni ditampilkan dalam dua format:
    \begin{itemize}
        \item \textbf{Tabel} (Desktop) - Menampilkan ID, Nama Klien, Instansi, Umpan Balik, Tanggal, Aksi
        \item \textbf{Card} (Mobile) - Format card yang lebih responsif
    \end{itemize}
    \item Terdapat fitur pagination untuk navigasi data
\end{enumerate}

\begin{figure}[h]
    \centering
    \includegraphics[width=0.9\textwidth]{images/testimoni-list.png}
    \caption{Tampilan Daftar Testimoni}
\end{figure}

\subsection{Mencari Testimoni}

\begin{enumerate}
    \item Gunakan kolom pencarian di bagian atas
    \item Ketik nama klien atau kata kunci dari testimoni
    \item Hasil akan difilter secara otomatis
\end{enumerate}

\subsection{Menambah Testimoni Baru}

\begin{enumerate}
    \item Klik tombol \textbf{"Tambah Testimoni"}
    \item Isi form dengan data:
    \begin{itemize}
        \item \textbf{Nama Klien*} - Nama pemberi testimoni
        \item \textbf{Instansi*} - Nama perusahaan/organisasi
        \item \textbf{Umpan Balik*} - Isi testimoni (maksimal 500 karakter)
        \item \textbf{Tanggal*} - Tanggal testimoni diberikan
    \end{itemize}
    \item Klik \textbf{"Tambah Testimoni"}
\end{enumerate}

\begin{figure}[h]
    \centering
    \includegraphics[width=0.7\textwidth]{images/testimoni-form-add.png}
    \caption{Form Tambah Testimoni}
\end{figure}

\subsection{Preview Testimoni}

Di bagian bawah halaman, terdapat section \textbf{"Pratinjau Testimoni"} yang menampilkan:
\begin{itemize}
    \item Tampilan testimoni seperti yang akan dilihat pengunjung
    \item Slider dengan pagination (maksimal 4 testimoni per halaman)
    \item Navigation dots dan tombol prev/next
\end{itemize}

\begin{figure}[h]
    \centering
    \includegraphics[width=0.9\textwidth]{images/testimoni-preview.png}
    \caption{Preview Testimoni di Halaman User}
\end{figure}

\subsection{Mengedit Testimoni}

\begin{enumerate}
    \item Klik tombol \textbf{"Edit"} pada baris testimoni
    \item Ubah data yang diperlukan
    \item Klik \textbf{"Perbarui Testimoni"}
\end{enumerate}

\subsection{Menghapus Testimoni}

\begin{enumerate}
    \item Klik tombol \textbf{"Hapus"}
    \item Konfirmasi penghapusan
    \item Klik \textbf{"Ya, Hapus"}
\end{enumerate}

\section{Visi \& Misi}

Fitur untuk mengelola visi dan misi perusahaan yang ditampilkan di halaman depan website.

\subsection{Tampilan Halaman Visi \& Misi}

Halaman dibagi menjadi dua kolom:
\begin{itemize}
    \item \textbf{Kolom Kiri} - Form Edit
    \item \textbf{Kolom Kanan} - Pratinjau
\end{itemize}

\begin{figure}[h]
    \centering
    \includegraphics[width=0.9\textwidth]{images/visi-misi-page.png}
    \caption{Halaman Manajemen Visi \& Misi}
\end{figure}

\subsection{Mengedit Visi}

\begin{enumerate}
    \item Pada kolom kiri, temukan field \textbf{"Visi"}
    \item Textarea akan berisi visi saat ini
    \item Edit teks visi sesuai kebutuhan
    \item Pratinjau akan otomatis berubah di kolom kanan
\end{enumerate}

\subsection{Mengelola Misi}

Misi dikelola dalam bentuk poin-poin:

\textbf{Menambah Poin Misi:}
\begin{enumerate}
    \item Klik tombol \textbf{"Tambah Poin Misi"}
    \item Field input baru akan muncul
    \item Ketik isi poin misi
\end{enumerate}

\textbf{Menghapus Poin Misi:}
\begin{enumerate}
    \item Klik tombol \textbf{"Hapus"} (ikon tempat sampah) di samping poin misi
    \item Poin akan langsung terhapus
    \item Minimal harus ada 1 poin misi
\end{enumerate}

\begin{figure}[h]
    \centering
    \includegraphics[width=0.7\textwidth]{images/visi-misi-form.png}
    \caption{Form Edit Visi \& Misi}
\end{figure}

\subsection{Menyimpan Perubahan}

\begin{enumerate}
    \item Setelah selesai mengedit visi dan misi
    \item Klik tombol \textbf{"Simpan Perubahan"}
    \item Sistem akan menyimpan data ke database
    \item Notifikasi sukses akan muncul
    \item Perubahan langsung terlihat di halaman frontend
\end{enumerate}

\section{Kontak}

Mengelola informasi kontak perusahaan yang ditampilkan di halaman Kontak.

\subsection{Informasi yang Dapat Dikelola}

\begin{itemize}
    \item Alamat perusahaan
    \item Nomor telepon/WhatsApp
    \item Email perusahaan
    \item URL Google Maps (untuk embed map)
\end{itemize}

\subsection{Mengedit Informasi Kontak}

\begin{enumerate}
    \item Klik menu \textbf{"Kontak"} pada sidebar
    \item Form akan menampilkan data kontak saat ini
    \item Edit field yang ingin diubah:
    \begin{itemize}
        \item \textbf{Alamat} - Alamat lengkap perusahaan
        \item \textbf{Nomor Telepon/WhatsApp} - Format: 08xx atau 62xxx
        \item \textbf{Email} - Email resmi perusahaan
        \item \textbf{Google Maps URL} - URL embed dari Google Maps
    \end{itemize}
    \item Klik \textbf{"Simpan"}
\end{enumerate}

\begin{figure}[h]
    \centering
    \includegraphics[width=0.7\textwidth]{images/kontak-form.png}
    \caption{Form Edit Kontak}
\end{figure}

\textbf{Cara Mendapatkan Google Maps Embed URL:}
\begin{enumerate}
    \item Buka Google Maps di browser
    \item Cari lokasi perusahaan
    \item Klik tombol "Share" atau "Bagikan"
    \item Pilih tab "Embed a map"
    \item Copy URL yang muncul
    \item Paste ke field "Google Maps URL"
\end{enumerate}

\section{Hero Section}

Mengelola gambar background slider yang ditampilkan di bagian atas halaman utama (Hero Section).

\subsection{Melihat Background yang Ada}

\begin{enumerate}
    \item Klik menu \textbf{"Hero Section"} pada sidebar
    \item Gambar background saat ini akan ditampilkan dalam grid
    \item Setiap gambar memiliki tombol hapus di pojok kanan atas
\end{enumerate}

\begin{figure}[h]
    \centering
    \includegraphics[width=0.9\textwidth]{images/hero-backgrounds.png}
    \caption{Daftar Background Hero Section}
\end{figure}

\subsection{Menambah Background Baru}

\begin{enumerate}
    \item Scroll ke section \textbf{"Tambah Gambar Baru"}
    \item Klik tombol \textbf{"Choose Files"} atau area upload
    \item Pilih satu atau beberapa gambar sekaligus
    \item Format yang didukung: JPG, PNG
    \item Ukuran maksimal per file: 100 MB
    \item Preview gambar akan muncul
\end{enumerate}

\textbf{Rekomendasi Gambar:}
\begin{itemize}
    \item Resolusi minimal: 1920x1080 pixels (Full HD)
    \item Aspect ratio: 16:9
    \item Gunakan gambar landscape (horizontal)
    \item Gambar yang kontras dan eye-catching
\end{itemize}

\subsection{Menghapus Background}

\begin{enumerate}
    \item Hover mouse ke gambar yang ingin dihapus
    \item Klik tombol \textbf{"X"} di pojok kanan atas
    \item Konfirmasi penghapusan
    \item Gambar akan terhapus dari server
\end{enumerate}

\subsection{Menyimpan Perubahan}

\begin{enumerate}
    \item Setelah upload atau hapus gambar
    \item Klik tombol \textbf{"Simpan Hero Section"}
    \item Sistem akan:
    \begin{itemize}
        \item Upload gambar baru ke CDN Cloudflare R2
        \item Menghapus gambar yang di-remove dari storage
        \item Update database
    \end{itemize}
    \item Notifikasi sukses akan muncul
\end{enumerate}

\begin{figure}[h]
    \centering
    \includegraphics[width=0.9\textwidth]{images/hero-save.png}
    \caption{Tombol Simpan Hero Section}
\end{figure}

\section{Riwayat Perusahaan}

Mengelola timeline sejarah perkembangan perusahaan.

\subsection{Melihat Daftar Riwayat}

\begin{enumerate}
    \item Klik menu \textbf{"Riwayat Perusahaan"} pada sidebar
    \item Daftar riwayat akan ditampilkan dalam bentuk tabel atau card
    \item Setiap item menampilkan:
    \begin{itemize}
        \item Tahun
        \item Judul event
        \item Deskripsi
        \item Gambar (jika ada)
        \item Tombol Edit dan Hapus
    \end{itemize}
\end{enumerate}

\begin{figure}[h]
    \centering
    \includegraphics[width=0.9\textwidth]{images/history-list.png}
    \caption{Daftar Riwayat Perusahaan}
\end{figure}

\subsection{Menambah Riwayat Baru}

\begin{enumerate}
    \item Klik tombol \textbf{"Tambah Riwayat"}
    \item Isi form dengan data:
    \begin{itemize}
        \item \textbf{Tahun*} - Tahun kejadian (format: YYYY)
        \item \textbf{Judul*} - Judul event/milestone
        \item \textbf{Deskripsi*} - Penjelasan detail kejadian
        \item \textbf{Gambar} - Upload foto terkait (opsional)
    \end{itemize}
    \item Klik \textbf{"Simpan"}
\end{enumerate}

\begin{figure}[h]
    \centering
    \includegraphics[width=0.7\textwidth]{images/history-form-add.png}
    \caption{Form Tambah Riwayat}
\end{figure}

\subsection{Mengedit Riwayat}

\begin{enumerate}
    \item Klik tombol \textbf{"Edit"} pada item riwayat
    \item Ubah data yang diperlukan
    \item Klik \textbf{"Simpan Perubahan"}
\end{enumerate}

\subsection{Menghapus Riwayat}

\begin{enumerate}
    \item Klik tombol \textbf{"Hapus"}
    \item Konfirmasi penghapusan
    \item Item riwayat akan dihapus permanen
\end{enumerate}

\section{Pengaturan Situs}

Mengelola label dan teks yang ditampilkan di berbagai section website.

\subsection{Label yang Dapat Dikonfigurasi}

\begin{table}[h]
\centering
\begin{tabular}{|p{5cm}|p{7cm}|}
\hline
\textbf{Section} & \textbf{Field yang Dapat Diubah} \\
\hline
Produk & Label section, Judul section \\
\hline
Our Client & Label section, Judul section \\
\hline
Testimoni & Label section, Judul section \\
\hline
Visi Misi & Label section, Judul section \\
\hline
Kontak & Label section, Judul section \\
\hline
Company History & Label section, Judul section \\
\hline
\end{tabular}
\caption{Konfigurasi Label Section}
\end{table}

\subsection{Mengubah Label Section}

\begin{enumerate}
    \item Klik menu \textbf{"Pengaturan Situs"} pada sidebar
    \item Temukan section yang ingin diubah labelnya
    \item Edit field label dan/atau title
    \item Contoh:
    \begin{itemize}
        \item Label Produk: "Produk Kami" atau "Our Products"
        \item Judul Produk: "Mesin Industri Berkualitas Tinggi"
    \end{itemize}
    \item Klik \textbf{"Simpan Pengaturan"}
\end{enumerate}

\begin{figure}[h]
    \centering
    \includegraphics[width=0.9\textwidth]{images/settings-page.png}
    \caption{Halaman Pengaturan Situs}
\end{figure}

\textbf{Tips Penamaan:}
\begin{itemize}
    \item Gunakan bahasa yang konsisten (Indonesia atau Inggris)
    \item Label sebaiknya singkat (2-3 kata)
    \item Title bisa lebih deskriptif (5-10 kata)
    \item Perhatikan SEO: gunakan kata kunci yang relevan
\end{itemize}

\section{Manajemen Admin (Khusus Super Admin)}

Fitur ini hanya dapat diakses oleh pengguna dengan role \textbf{Super Admin}.

\subsection{Melihat Daftar Admin}

\begin{enumerate}
    \item Klik menu \textbf{"Manajemen Admin"} pada sidebar
    \item Daftar admin akan ditampilkan dalam tabel
    \item Informasi yang ditampilkan:
    \begin{itemize}
        \item Username
        \item Role (Admin atau Super Admin)
        \item Tanggal dibuat
        \item Tombol Edit dan Hapus
    \end{itemize}
\end{enumerate}

\begin{figure}[h]
    \centering
    \includegraphics[width=0.9\textwidth]{images/admin-list.png}
    \caption{Daftar Administrator}
\end{figure}

\subsection{Menambah Admin Baru}

\begin{enumerate}
    \item Klik tombol \textbf{"Tambah Admin"}
    \item Isi form dengan data:
    \begin{itemize}
        \item \textbf{Username*} - Harus unik
        \item \textbf{Password*} - Minimal 8 karakter
        \item \textbf{Konfirmasi Password*} - Harus sama dengan password
        \item \textbf{Role*} - Pilih Admin atau Super Admin
    \end{itemize}
    \item Klik \textbf{"Tambah Admin"}
\end{enumerate}

% \begin{figure}[h]
%     \centering
%     \includegraphics[width=0.7\textwidth]{images/admin-form-add.png}
%     \caption{Form Tambah Administrator}
% \end{figure} % Image not found

\textbf{Perbedaan Role:}
\begin{itemize}
    \item \textbf{Admin} - Dapat mengelola semua konten kecuali manajemen admin
    \item \textbf{Super Admin} - Memiliki akses penuh termasuk manajemen admin
\end{itemize}

\subsection{Mengedit Admin}

\begin{enumerate}
    \item Klik tombol \textbf{"Edit"} pada baris admin
    \item Ubah data yang diperlukan
    \item Untuk mengganti password:
    \begin{itemize}
        \item Isi field "Password Baru"
        \item Isi field "Konfirmasi Password Baru"
        \item Biarkan kosong jika tidak ingin mengubah password
    \end{itemize}
    \item Klik \textbf{"Simpan Perubahan"}
\end{enumerate}

\subsection{Menghapus Admin}

\begin{enumerate}
    \item Klik tombol \textbf{"Hapus"}
    \item Konfirmasi penghapusan
    \item Admin akan dihapus dari sistem
\end{enumerate}

\textbf{Catatan Keamanan:}
\begin{itemize}
    \item Tidak dapat menghapus akun sendiri yang sedang login
    \item Minimal harus ada 1 Super Admin di sistem
\end{itemize}

% =====================
% CHAPTER 4: FITUR HALAMAN USER
% =====================
\chapter{Fitur-Fitur Halaman User}

Bagian ini menjelaskan fitur-fitur yang dapat diakses oleh pengunjung website (tanpa login).

\section{Navigasi Website}

\subsection{Menu Navbar}

Navbar website bersifat \textit{sticky} (selalu terlihat di bagian atas saat scroll).

\textbf{Menu yang Tersedia:}
\begin{itemize}
    \item \textbf{Home} - Kembali ke halaman beranda
    \item \textbf{Produk} - Menuju section produk
    \item \textbf{Klien} - Menuju section klien
    \item \textbf{Testimoni} - Menuju section testimoni
    \item \textbf{Tentang} - Menuju section visi misi dan company history
    \item \textbf{Kontak} - Menuju section kontak
\end{itemize}

\begin{figure}[h]
    \centering
    \includegraphics[width=0.9\textwidth]{images/navbar-user.png}
    \caption{Menu Navigasi Website}
\end{figure}

\subsection{Navigasi Mobile}

Pada perangkat mobile (tablet dan smartphone):
\begin{itemize}
    \item Menu akan berubah menjadi hamburger icon
    \item Klik icon untuk membuka sidebar menu
    \item Menu ditampilkan secara vertikal
    \item Klik di luar sidebar untuk menutup
\end{itemize}

\begin{figure}[h]
    \centering
    \includegraphics[width=0.5\textwidth]{images/navbar-mobile.png}
    \caption{Menu Mobile}
\end{figure}

\section{Halaman Beranda}

\subsection{Hero Section}

Bagian paling atas website yang menampilkan:
\begin{itemize}
    \item Background slider dengan multiple gambar
    \item Overlay gradient untuk readability
    \item Judul dan subtitle perusahaan
    \item Call-to-action button
    \item Animasi fade transition antar gambar
\end{itemize}

\begin{figure}[h]
    \centering
    \includegraphics[width=0.9\textwidth]{images/hero-section.png}
    \caption{Hero Section}
\end{figure}

\textbf{Fitur Hero Section:}
\begin{itemize}
    \item Auto-slide setiap 5 detik
    \item Smooth transition effect
    \item Responsive di semua perangkat
    \item Optimized image loading (WebP format)
\end{itemize}

\subsection{Section Produk}

Menampilkan produk-produk mesin industri dengan fitur:

\textbf{Fitur Utama:}
\begin{itemize}
    \item Tabs untuk setiap kategori produk
    \item Gambar produk utama (besar)
    \item Galeri gambar produk (thumbnail)
    \item Informasi detail produk:
    \begin{itemize}
        \item Nama produk
        \item Harga (format Rupiah)
        \item Deskripsi lengkap
    \end{itemize}
    \item Navigation arrows untuk switch produk
    \item Page dots indicator
\end{itemize}

\begin{figure}[h]
    \centering
    \includegraphics[width=0.9\textwidth]{images/produk-section.png}
    \caption{Section Produk}
\end{figure}

\textbf{Cara Menggunakan:}
\begin{enumerate}
    \item Klik tab nama produk untuk melihat detail
    \item Klik thumbnail galeri untuk melihat gambar besar
    \item Gunakan arrow kiri/kanan untuk navigasi produk
    \item Klik dots untuk jump ke produk tertentu
\end{enumerate}

\subsection{Section Our Client}

Menampilkan logo-logo klien yang telah bekerja sama.

\textbf{Fitur:}
\begin{itemize}
    \item Grid layout (4 kolom di desktop, responsive di mobile)
    \item Logo klien dengan border card
    \item Hover effect pada setiap card
    \item Slider pagination jika klien lebih dari 4
    \item Auto-slide setiap 3 detik
\end{itemize}

\begin{figure}[h]
    \centering
    \includegraphics[width=0.9\textwidth]{images/client-section.png}
    \caption{Section Our Client}
\end{figure}

\subsection{Section Testimoni}

Menampilkan testimoni dari pelanggan yang puas.

\textbf{Fitur:}
\begin{itemize}
    \item Card-based layout
    \item Avatar initial dari nama klien
    \item Nama klien dan institusi
    \item Isi testimoni dengan format italic
    \item Tanggal testimoni
    \item Slider dengan max 4 testimoni per slide
    \item Navigation dots dan arrow buttons
\end{itemize}

\begin{figure}[h]
    \centering
    \includegraphics[width=0.9\textwidth]{images/testimoni-section.png}
    \caption{Section Testimoni}
\end{figure}

\subsection{Section Visi \& Misi}

Menampilkan visi dan misi perusahaan.

\textbf{Layout:}
\begin{itemize}
    \item Dua kolom (desktop) / vertical stack (mobile)
    \item Kolom kiri: Visi dengan icon eye
    \item Kolom kanan: Misi dengan bullet points
    \item Background gradient subtle
    \item Icon yang menarik untuk visual appeal
\end{itemize}

\begin{figure}[h]
    \centering
    \includegraphics[width=0.9\textwidth]{images/visi-misi-section.png}
    \caption{Section Visi \& Misi}
\end{figure}

\subsection{Section Company History}

Timeline sejarah perkembangan perusahaan.

\textbf{Fitur:}
\begin{itemize}
    \item Vertical timeline dengan connector line
    \item Timeline dots untuk setiap milestone
    \item Tahun ditampilkan di sebelah kiri
    \item Card informasi di sebelah kanan
    \item Gambar pendukung (jika ada)
    \item Animasi on-scroll (fade-in effect)
\end{itemize}

\begin{figure}[h]
    \centering
    \includegraphics[width=0.9\textwidth]{images/history-section.png}
    \caption{Section Company History}
\end{figure}

\section{Halaman Kontak}

\subsection{Informasi Kontak}

Menampilkan 3 card informasi kontak:

\begin{enumerate}
    \item \textbf{Alamat}
    \begin{itemize}
        \item Icon map marker
        \item Alamat lengkap perusahaan
    \end{itemize}

    \item \textbf{WhatsApp}
    \begin{itemize}
        \item Icon WhatsApp
        \item Nomor WhatsApp (clickable)
        \item Klik untuk langsung chat
    \end{itemize}

    \item \textbf{Email}
    \begin{itemize}
        \item Icon envelope
        \item Email perusahaan (clickable)
        \item Klik untuk buka email client
    \end{itemize}
\end{enumerate}

\begin{figure}[h]
    \centering
    \includegraphics[width=0.9\textwidth]{images/kontak-info.png}
    \caption{Informasi Kontak}
\end{figure}

\subsection{Form WhatsApp}

Fitur baru yang memudahkan pengunjung untuk langsung menghubungi via WhatsApp.

\textbf{Komponen Form:}
\begin{itemize}
    \item Input Nama Lengkap (required)
    \item Textarea Pesan (required)
    \item Button "Chat via WhatsApp"
\end{itemize}

\begin{figure}[h]
    \centering
    \includegraphics[width=0.9\textwidth]{images/whatsapp-form.png}
    \caption{Form WhatsApp}
\end{figure}

\textbf{Cara Menggunakan:}
\begin{enumerate}
    \item Isi nama lengkap Anda
    \item Tulis pesan yang ingin disampaikan
    \item Klik tombol \textbf{"Chat via WhatsApp"}
    \item WhatsApp Web/App akan terbuka otomatis
    \item Pesan sudah terformat dengan nama Anda:
    \begin{verbatim}
Halo, nama saya *[Nama Anda]*

[Isi Pesan Anda]
    \end{verbatim}
    \item Tinggal klik Send di WhatsApp
\end{enumerate}

\textbf{Keuntungan Fitur Ini:}
\begin{itemize}
    \item Tidak perlu mengetik ulang di WhatsApp
    \item Pesan sudah terformat rapi
    \item Langsung terhubung dengan nomor perusahaan
    \item Mudah digunakan dari desktop atau mobile
\end{itemize}

\subsection{Google Maps}

Section paling bawah menampilkan embedded Google Maps:

\textbf{Fitur:}
\begin{itemize}
    \item Interactive map dengan zoom dan pan
    \item Marker lokasi perusahaan
    \item Dapat dibuka di Google Maps app
    \item Responsive iframe
    \item Rounded corners untuk estetika
\end{itemize}

\begin{figure}[h]
    \centering
    \includegraphics[width=0.9\textwidth]{images/google-maps.png}
    \caption{Google Maps Embed}
\end{figure}

\section{Footer Website}

Footer ditampilkan di bagian paling bawah setiap halaman.

\textbf{Informasi yang Ditampilkan:}
\begin{itemize}
    \item Logo perusahaan
    \item Nama perusahaan
    \item Alamat singkat
    \item Link kontak
    \item Link media sosial (jika ada)
    \item Copyright notice
\end{itemize}

\begin{figure}[h]
    \centering
    \includegraphics[width=0.9\textwidth]{images/footer.png}
    \caption{Footer Website}
\end{figure}

\section{WhatsApp Floating Button}

Tombol WhatsApp yang selalu terlihat di pojok kanan bawah layar.

\textbf{Fitur:}
\begin{itemize}
    \item Fixed position (sticky)
    \item Icon WhatsApp hijau
    \item Pulse animation untuk menarik perhatian
    \item Hover effect
    \item Klik untuk langsung chat WhatsApp
\end{itemize}

\begin{figure}[h]
    \centering
    \includegraphics[width=0.3\textwidth]{images/wa-floating-button.png}
    \caption{WhatsApp Floating Button}
\end{figure}

\textbf{Cara Menggunakan:}
\begin{enumerate}
    \item Klik tombol WhatsApp floating
    \item WhatsApp akan terbuka dengan pesan default
    \item User dapat langsung chat dengan perusahaan
\end{enumerate}

% =====================
% CHAPTER 5: RESPONSIVE DESIGN
% =====================
\chapter{Responsive Design}

Website ini dirancang untuk tampil optimal di berbagai ukuran layar.

\section{Breakpoint Responsive}

\begin{table}[h]
\centering
\begin{tabular}{|l|l|p{6cm}|}
\hline
\textbf{Device} & \textbf{Breakpoint} & \textbf{Keterangan} \\
\hline
Mobile & < 640px & 1 kolom, menu hamburger \\
\hline
Tablet & 640px - 1024px & 2 kolom, menu partial \\
\hline
Desktop & > 1024px & Full layout, semua fitur visible \\
\hline
\end{tabular}
\caption{Responsive Breakpoints}
\end{table}

\section{Fitur Responsive Admin}

\subsection{Mobile Sidebar}

Pada layar mobile (< 1024px):
\begin{itemize}
    \item Sidebar tersembunyi secara default
    \item Muncul tombol hamburger di header
    \item Klik hamburger untuk toggle sidebar
    \item Overlay hitam saat sidebar terbuka
    \item Klik overlay untuk menutup sidebar
    \item Auto-close saat memilih menu
\end{itemize}

% \begin{figure}[h]
%     \centering
%     \includegraphics[width=0.9\textwidth]{images/admin-mobile-sidebar.png}
%     \caption{Mobile Sidebar Admin}
% \end{figure} % Image not found

\subsection{Responsive Tables}

Tabel pada halaman admin (seperti Testimoni) memiliki dua mode:

\textbf{Desktop (> 1024px):}
\begin{itemize}
    \item Tampilan tabel tradisional
    \item Semua kolom terlihat
    \item Sortable columns
\end{itemize}

\textbf{Mobile (< 1024px):}
\begin{itemize}
    \item Berubah menjadi card view
    \item Setiap row menjadi card
    \item Informasi disusun vertikal
    \item Action buttons full-width
\end{itemize}

\begin{figure}[h]
    \centering
    \includegraphics[width=0.9\textwidth]{images/responsive-table.png}
    \caption{Responsive Table (Desktop vs Mobile)}
\end{figure}

\subsection{Form Responsive}

Semua form input di admin responsive:
\begin{itemize}
    \item Input fields full-width di mobile
    \item Button full-width di mobile
    \item Spacing disesuaikan per device
    \item Modal dialog menyesuaikan lebar layar
\end{itemize}

\section{Fitur Responsive Frontend}

\subsection{Hero Section Responsive}

\begin{itemize}
    \item Background image menyesuaikan layar
    \item Text size scalable (text-2xl sm:text-4xl lg:text-6xl)
    \item Padding dan margin responsive
    \item CTA button menyesuaikan ukuran
\end{itemize}

\subsection{Product Grid Responsive}

\begin{itemize}
    \item Mobile: 1 kolom
    \item Tablet: 2 kolom
    \item Desktop: Layout khusus dengan gambar besar dan info samping
\end{itemize}

\subsection{Client Grid Responsive}

\begin{itemize}
    \item Mobile: 2 kolom
    \item Tablet: 3 kolom
    \item Desktop: 4 kolom
    \item Max-width untuk card consistency
\end{itemize}

\subsection{WhatsApp Form Responsive}

\begin{itemize}
    \item Mobile: Stack vertikal (form di atas, dekorasi hidden)
    \item Desktop: Side-by-side (form kiri, dekorasi kanan)
    \item Padding menyesuaikan: p-8 sm:p-12
\end{itemize}

% =====================
% CHAPTER 6: TROUBLESHOOTING
% =====================
\chapter{Troubleshooting}

Bagian ini membantu menyelesaikan masalah umum yang mungkin terjadi.

\section{Masalah Login Admin}

\subsection{Tidak Bisa Login}

\textbf{Gejala:} Setelah memasukkan username dan password, muncul pesan error atau tidak bisa masuk.

\textbf{Solusi:}
\begin{enumerate}
    \item Pastikan username dan password benar (case-sensitive)
    \item Cek koneksi internet Anda
    \item Clear browser cache dan cookies
    \item Coba browser lain (Chrome, Firefox, Edge)
    \item Pastikan server backend tidak down
    \item Hubungi Super Admin untuk reset password
\end{enumerate}

\subsection{Lupa Password}

\textbf{Solusi:}
\begin{enumerate}
    \item Hubungi Super Admin
    \item Super Admin dapat reset password Anda dari menu Manajemen Admin
    \item Login dengan password baru
    \item Segera ganti password Anda di halaman profile
\end{enumerate}

\section{Masalah Upload File}

\subsection{Gagal Upload Gambar}

\textbf{Gejala:} Error saat upload gambar produk, logo, atau background.

\textbf{Penyebab dan Solusi:}

\begin{table}[h]
\centering
\begin{tabular}{|p{5cm}|p{7cm}|}
\hline
\textbf{Penyebab} & \textbf{Solusi} \\
\hline
File terlalu besar & Kompres gambar dengan tool online (TinyPNG, Compressor.io) \\
\hline
Format tidak didukung & Gunakan format JPG, PNG, atau WebP \\
\hline
Koneksi internet lambat & Tunggu hingga upload selesai, jangan refresh page \\
\hline
Server storage penuh & Hubungi administrator server \\
\hline
\end{tabular}
\caption{Troubleshooting Upload File}
\end{table}

\subsection{Gambar Tidak Muncul Setelah Upload}

\textbf{Solusi:}
\begin{enumerate}
    \item Refresh halaman (Ctrl+F5 atau Cmd+Shift+R)
    \item Clear browser cache
    \item Cek apakah file benar-benar terupload di server
    \item Pastikan URL CDN aktif dan accessible
    \item Cek console browser untuk error messages
\end{enumerate}

\section{Masalah Tampilan Website}

\subsection{Tampilan Berantakan / Layout Rusak}

\textbf{Solusi:}
\begin{enumerate}
    \item Hard refresh browser (Ctrl+Shift+R)
    \item Clear cache dan cookies
    \item Update browser ke versi terbaru
    \item Cek apakah CSS/JS file berhasil dimuat (F12 > Network tab)
    \item Disable browser extensions yang mungkin mengganggu
\end{enumerate}

\subsection{Website Lambat Loading}

\textbf{Penyebab dan Solusi:}
\begin{itemize}
    \item \textbf{Gambar terlalu besar:} Kompres dan optimasi gambar
    \item \textbf{Server lambat:} Hubungi hosting provider
    \item \textbf{Koneksi internet lambat:} Cek speed test internet
    \item \textbf{CDN issue:} Pastikan Cloudflare R2 berfungsi normal
\end{itemize}

\subsection{Responsive Tidak Berfungsi}

\textbf{Gejala:} Tampilan mobile tidak responsive.

\textbf{Solusi:}
\begin{enumerate}
    \item Pastikan viewport meta tag ada di HTML
    \item Clear cache browser
    \item Test dengan device atau emulator lain
    \item Update browser mobile
\end{enumerate}

\section{Masalah Fungsionalitas}

\subsection{Form WhatsApp Tidak Berfungsi}

\textbf{Gejala:} Klik tombol "Chat via WhatsApp" tidak membuka WhatsApp.

\textbf{Solusi:}
\begin{enumerate}
    \item Pastikan nomor WhatsApp di database benar (format 62xxx)
    \item Cek browser pop-up blocker (allow pop-up)
    \item Pastikan WhatsApp terinstall atau akses dari WhatsApp Web
    \item Test dengan browser lain
\end{enumerate}

\subsection{Sidebar Admin Tidak Bisa Ditutup (Mobile)}

\textbf{Solusi:}
\begin{enumerate}
    \item Klik di luar area sidebar (overlay hitam)
    \item Klik tombol X di dalam sidebar
    \item Pilih salah satu menu (auto-close)
    \item Refresh halaman jika masih stuck
\end{enumerate}

\subsection{Data Tidak Tersimpan}

\textbf{Gejala:} Setelah klik "Simpan", data tidak berubah.

\textbf{Solusi:}
\begin{enumerate}
    \item Cek apakah ada pesan error di halaman
    \item Pastikan semua field required terisi
    \item Cek koneksi internet
    \item Buka F12 > Console untuk melihat error
    \item Coba logout dan login kembali
    \item Hubungi administrator jika masalah berlanjut
\end{enumerate}

\section{Masalah Keamanan}

\subsection{Akun Terkunci}

\textbf{Gejala:} Tidak bisa login setelah beberapa kali salah password.

\textbf{Solusi:}
\begin{enumerate}
    \item Tunggu 15-30 menit
    \item Hubungi Super Admin untuk unlock akun
    \item Reset password melalui Super Admin
\end{enumerate}

\subsection{Suspicious Activity}

\textbf{Gejala:} Melihat perubahan data yang tidak Anda lakukan.

\textbf{Tindakan:}
\begin{enumerate}
    \item Segera logout dari semua device
    \item Ganti password immediately
    \item Laporkan ke Super Admin
    \item Cek activity log jika tersedia
\end{enumerate}

% =====================
% CHAPTER 7: KEAMANAN
% =====================
\chapter{Keamanan dan Best Practices}

\section{Kebijakan Keamanan}

\subsection{Password Policy}

\textbf{Persyaratan Password:}
\begin{itemize}
    \item Minimal 8 karakter
    \item Kombinasi huruf besar dan kecil
    \item Mengandung angka
    \item Disarankan mengandung karakter spesial (!@\#\$\%\^\&*)
    \item Tidak menggunakan kata yang mudah ditebak (nama, tanggal lahir)
    \item Diganti secara berkala (3-6 bulan sekali)
\end{itemize}

\subsection{Enkripsi}

\begin{itemize}
    \item Password disimpan dengan hash encryption (bcrypt)
    \item Data sensitif di-encrypt saat transit (HTTPS)
    \item Session token menggunakan secure cookies
\end{itemize}

\subsection{Hak Akses}

\begin{table}[h]
\centering
\begin{tabular}{|l|c|c|}
\hline
\textbf{Fitur} & \textbf{Admin} & \textbf{Super Admin} \\
\hline
View Dashboard & \checkmark & \checkmark \\
Manage Produk & \checkmark & \checkmark \\
Manage Klien & \checkmark & \checkmark \\
Manage Testimoni & \checkmark & \checkmark \\
Manage Konten & \checkmark & \checkmark \\
Manage Admin & & \checkmark \\
\hline
\end{tabular}
\caption{Hak Akses Berdasarkan Role}
\end{table}

\section{Best Practices untuk Administrator}

\subsection{Keamanan Akun}

\begin{enumerate}
    \item \textbf{Jangan Bagikan Kredensial}
    \begin{itemize}
        \item Jangan share username/password ke siapapun
        \item Setiap admin harus punya akun sendiri
    \end{itemize}

    \item \textbf{Selalu Logout}
    \begin{itemize}
        \item Logout setelah selesai bekerja
        \item Jangan tinggalkan komputer dengan session aktif
    \end{itemize}

    \item \textbf{Gunakan Jaringan Aman}
    \begin{itemize}
        \item Hindari WiFi publik untuk login admin
        \item Gunakan VPN jika perlu akses dari luar
    \end{itemize}

    \item \textbf{Update Browser}
    \begin{itemize}
        \item Selalu gunakan browser versi terbaru
        \item Enable auto-update jika memungkinkan
    \end{itemize}
\end{enumerate}

\subsection{Pengelolaan Konten}

\begin{enumerate}
    \item \textbf{Backup Sebelum Edit Besar}
    \begin{itemize}
        \item Screenshot atau catat data penting
        \item Koordinasi dengan admin lain sebelum perubahan besar
    \end{itemize}

    \item \textbf{Validasi Data Input}
    \begin{itemize}
        \item Cek ejaan dan grammar
        \item Pastikan gambar berkualitas baik
        \item Verifikasi link dan kontak
    \end{itemize}

    \item \textbf{Test Setelah Update}
    \begin{itemize}
        \item Buka website sebagai user setelah update
        \item Cek apakah perubahan terlihat
        \item Test di mobile dan desktop
    \end{itemize}
\end{enumerate}

\subsection{Pengelolaan File}

\begin{enumerate}
    \item \textbf{Penamaan File Konsisten}
    \begin{itemize}
        \item Gunakan nama file yang descriptive
        \item Hindari spasi, gunakan dash atau underscore
        \item Contoh: mesin-bubut-cnc-2024.jpg
    \end{itemize}

    \item \textbf{Optimasi Gambar}
    \begin{itemize}
        \item Kompres gambar sebelum upload
        \item Gunakan format WebP untuk performance
        \item Resize ke ukuran yang sesuai
    \end{itemize}

    \item \textbf{Organisasi File}
    \begin{itemize}
        \item Hapus file yang tidak digunakan
        \item Dokumentasikan perubahan penting
    \end{itemize}
\end{enumerate}

\section{Compliance dan Privacy}

\subsection{Data Privacy}

\begin{itemize}
    \item Data pelanggan dijaga kerahasiaannya
    \item Tidak membagikan informasi kontak tanpa izin
    \item Testimoni dipublikasi dengan persetujuan
\end{itemize}

\subsection{Content Policy}

\begin{itemize}
    \item Konten harus akurat dan tidak menyesatkan
    \item Gambar dan teks bebas dari plagiarisme
    \item Tidak mengandung SARA atau konten ilegal
    \item Sesuai dengan branding perusahaan
\end{itemize}

% =====================
% CHAPTER 8: LAMPIRAN
% =====================
\chapter{Lampiran}

\section{Glossary}

\begin{description}
    \item[Admin] Pengguna dengan hak akses untuk mengelola konten website
    \item[Backend] Sistem di sisi server yang mengelola data dan logika aplikasi
    \item[CDN] Content Delivery Network - Sistem untuk distribusi file statis
    \item[CRUD] Create, Read, Update, Delete - Operasi dasar database
    \item[Frontend] Tampilan website yang dilihat oleh pengunjung
    \item[Responsive] Desain yang menyesuaikan dengan ukuran layar perangkat
    \item[Session] Period waktu dimana user tetap login di sistem
    \item[Super Admin] Admin dengan hak akses penuh termasuk manajemen user
    \item[Upload] Proses mengirim file dari komputer lokal ke server
    \item[URL] Uniform Resource Locator - Alamat website
\end{description}

\section{Informasi Kontak}

\subsection{Support Teknis}

Jika mengalami masalah teknis atau memerlukan bantuan, hubungi:

\begin{itemize}
    \item \textbf{Email Support:} support@suryakencana.co.id
    \item \textbf{WhatsApp:} +62 812-3456-7890
    \item \textbf{Jam Operasional:} Senin - Jumat, 09:00 - 17:00 WIB
\end{itemize}

\subsection{Administrator Website}

\begin{itemize}
    \item \textbf{IT Department:} it@suryakencana.co.id
    \item \textbf{Marketing Department:} marketing@suryakencana.co.id
\end{itemize}

\section{Riwayat Perubahan Dokumen}

\begin{table}[h]
\centering
\begin{tabular}{|c|l|p{7cm}|}
\hline
\textbf{Versi} & \textbf{Tanggal} & \textbf{Perubahan} \\
\hline
1.0 & 20 Des 2024 & Rilis awal user manual lengkap \\
\hline
1.1 & TBD & Update fitur baru (jika ada) \\
\hline
\end{tabular}
\caption{Version History}
\end{table}

\section{Referensi}

\subsection{Link Penting}

\begin{itemize}
    \item Website Frontend: \url{https://becompro.app.fizualstd.my.id/}
    \item Admin Dashboard: \url{https://becompro.app.fizualstd.my.id/admin/dashboard}
    \item Laravel Documentation: \url{https://laravel.com/docs}
    \item Vue.js Documentation: \url{https://vuejs.org/guide/}
    \item Tailwind CSS: \url{https://tailwindcss.com/docs}
\end{itemize}

\subsection{Tools Pendukung}

\textbf{Image Optimization:}
\begin{itemize}
    \item TinyPNG - \url{https://tinypng.com/}
    \item Compressor.io - \url{https://compressor.io/}
    \item Squoosh - \url{https://squoosh.app/}
\end{itemize}

\textbf{Testing:}
\begin{itemize}
    \item Google PageSpeed Insights - \url{https://pagespeed.web.dev/}
    \item GTmetrix - \url{https://gtmetrix.com/}
    \item Mobile-Friendly Test - \url{https://search.google.com/test/mobile-friendly}
\end{itemize}

\section{FAQ (Frequently Asked Questions)}

\subsection{Pertanyaan Umum User}

\textbf{Q: Apakah saya perlu registrasi untuk melihat website?}\\
A: Tidak, semua konten website dapat diakses tanpa registrasi atau login.

\textbf{Q: Bagaimana cara memesan produk?}\\
A: Silakan hubungi kami via WhatsApp (gunakan form WhatsApp di halaman Kontak) atau klik tombol floating WhatsApp di pojok kanan bawah.

\textbf{Q: Apakah bisa request katalog produk?}\\
A: Ya, silakan hubungi kami via WhatsApp atau email untuk mendapatkan katalog lengkap.

\subsection{Pertanyaan Umum Admin}

\textbf{Q: Berapa lama waktu yang dibutuhkan untuk update konten terlihat di website?}\\
A: Perubahan akan terlihat langsung setelah disimpan. Namun, terkadang perlu refresh browser untuk melihat perubahan.

\textbf{Q: Apakah bisa menambah admin baru?}\\
A: Ya, tetapi hanya Super Admin yang dapat menambah admin baru melalui menu Manajemen Admin.

\textbf{Q: Bagaimana jika lupa password?}\\
A: Hubungi Super Admin untuk reset password Anda.

\textbf{Q: Apakah ada limit jumlah produk/klien/testimoni?}\\
A: Tidak ada limit hard-coded, tetapi pertimbangkan performance website jika data terlalu banyak.

\textbf{Q: Bisa upload video untuk produk?}\\
A: Saat ini hanya mendukung gambar. Untuk video, bisa gunakan link YouTube di deskripsi.

\section{Catatan Penutup}

Terima kasih telah menggunakan sistem Website Company Profile PT. Surya Kencana Gemilang Teknik. Dokumen ini akan terus diperbarui seiring dengan penambahan fitur baru atau perubahan pada sistem.

Jika ada pertanyaan, saran, atau masukan untuk perbaikan user manual ini, silakan hubungi tim IT Department.

\vspace{1cm}

\textbf{Semoga bermanfaat!}

\vspace{2cm}

\begin{center}
\line(1,0){250}\\
\textit{Dokumen ini adalah properti PT. Surya Kencana Gemilang Teknik}\\
\textit{dan bersifat rahasia. Dilarang menggandakan tanpa izin.}
\end{center}

\end{document}
