\documentclass[12pt,a4paper]{report}

% Packages
\usepackage[utf8]{inputenc}
\usepackage[bahasa]{babel}
\usepackage[T1]{fontenc}
\usepackage{graphicx} 
\usepackage{geometry}
\usepackage{hyperref}
\usepackage{float}
\usepackage{titlesec}
\usepackage{fancyhdr}
\usepackage{listings}
\usepackage{xcolor}
\usepackage{tabularx}
\usepackage{longtable}
\usepackage{booktabs}
\usepackage{enumitem}
\usepackage{tcolorbox}
\usepackage{amssymb}
\usepackage{menukeys} % Untuk styling tombol/menu

% Configuration
\geometry{
    top=2.5cm,
    bottom=2.5cm,
    left=3cm,
    right=2.5cm
}

\graphicspath{{./images/}}

\hypersetup{
    colorlinks=true,
    linkcolor=black,
    filecolor=magenta,
    urlcolor=cyan,
    pdftitle={User Manual - PT. Surya Kencana Gemilang Teknik},
    pdfauthor={Janu, Fahim, Hafidz, Akmal},
}

% Header & Footer
\pagestyle{fancy}
\fancyhf{}
\rhead{User Manual Website}
\lhead{PT. Surya Kencana Gemilang Teknik}
\cfoot{\thepage}

% Title Format
\titleformat{\chapter}[display]
  {\normalfont\Huge\bfseries}{\chaptertitlename\ \thechapter}{20pt}{\Huge}

% Custom boxes
\newtcolorbox{infobox}[1][]{
  colback=blue!5!white,
  colframe=blue!75!black,
  fonttitle=\bfseries,
  title=#1
}

\newtcolorbox{warningbox}[1][]{
  colback=red!5!white,
  colframe=red!75!black,
  fonttitle=\bfseries,
  title=#1
}

\newtcolorbox{tipbox}[1][]{
  colback=green!5!white,
  colframe=green!75!black,
  fonttitle=\bfseries,
  title=#1
}

\begin{document}

%----------------------------------------------------------------------------------------
%   TITLE PAGE
%----------------------------------------------------------------------------------------
\begin{titlepage}
    \centering
    \vspace*{1cm}

    {\Huge \textbf{PANDUAN PENGGUNA SISTEM}}\\[0.5cm]
    {\LARGE \textbf{(USER MANUAL)}}\\[1.5cm]
        \begin{figure}[H]
        \centering
        \includegraphics[width=0.4\textwidth]{logo.png}
        \label{fig:logo-company}
    \end{figure}


    {\Large \textbf{WEBSITE COMPANY PROFILE}}\\[0.5cm]
    {\Huge \textbf{PT. SURYA KENCANA}}\\[0.2cm]
    {\Huge \textbf{GEMILANG TEKNIK}}\\[2cm]

    \textbf{Tanggal: \today}\\[2cm]

    \textbf{Disusun Oleh Tim Pengembang:}\\
    \vspace{0.5cm}
    \begin{tabular}{ll}
        \textbf{Project Manager:} & Janu \\
        \textbf{UI/UX Designer:} & Fahim \\
        \textbf{Frontend Developer:} & Hafidz \\
        \textbf{Backend Developer:} & Akmal \\
    \end{tabular}\\
    \vspace{0.5cm}
    \textbf{Dosen Pembimbing:} Pak Galih\\[1cm]

\end{titlepage}

%----------------------------------------------------------------------------------------
%   TABLE OF CONTENTS
%----------------------------------------------------------------------------------------
\tableofcontents
\listoffigures
\newpage

%----------------------------------------------------------------------------------------
%   BAB 1: PENDAHULUAN
%----------------------------------------------------------------------------------------
\chapter{Pendahuluan}

\section{Pengertian Website Profil Perusahaan PT. Surya Kencana Gemilang Teknik}
Website Profil Perusahaan PT. Surya Kencana Gemilang Teknik adalah platform digital yang dirancang khusus untuk memperkenalkan, mempromosikan, dan menyajikan informasi lengkap tentang perusahaan kami kepada calon klien dan mitra bisnis. Website ini berfungsi sebagai representasi digital dari identitas perusahaan, menampilkan produk-produk unggulan, sejarah perusahaan, portofolio, serta informasi kontak yang mudah diakses oleh pengunjung dari seluruh dunia.

Platform ini dirancang dengan pendekatan modern dan responsif, memungkinkan pengunjung untuk mengakses informasi perusahaan dengan cepat dan mudah melalui berbagai perangkat, baik desktop maupun mobile. Website ini juga dilengkapi dengan sistem manajemen konten (CMS) yang memungkinkan tim internal untuk mengelola dan memperbarui informasi secara mandiri tanpa ketergantungan pada pihak eksternal.

\section{Tentang Dokumen Ini}
Dokumen ini adalah panduan teknis mendalam untuk penggunaan Website Company Profile PT. Surya Kencana Gemilang Teknik. Panduan ini mencakup instruksi langkah demi langkah untuk setiap fitur yang tersedia di panel administrasi (CMS) dan penjelasan mengenai antarmuka publik. Dokumen ini ditujukan untuk administrator sistem, staf marketing, dan pihak-pihak terkait lainnya yang bertanggung jawab atas pengelolaan konten website.

\section{Deskripsi Sistem}
Website ini dikembangkan menggunakan teknologi modern untuk menjamin kecepatan dan kemudahan penggunaan:
\begin{itemize}
    \item \textbf{Frontend}: Vue.js dengan TypeScript dan Tailwind CSS.
    \item \textbf{Backend}: Laravel API.
\end{itemize}

\section{Hak Akses Pengguna}
Sistem membedakan dua tingkat akses administrator:
\begin{enumerate}
    \item \textbf{Admin}: Memiliki akses penuh untuk mengelola konten (Produk, Klien, Testimoni, Hero, dll).
    \item \textbf{Super Admin}: Memiliki semua akses Admin ditambah hak khusus untuk \textbf{Manajemen Admin} (menambah, mengedit, atau menghapus akun admin lain).
\end{enumerate}

\section{Persyaratan Sistem}
Untuk menggunakan sistem dengan optimal, pastikan perangkat Anda memenuhi persyaratan berikut:
\begin{itemize}
    \item Browser modern (Chrome, Firefox, Safari, Edge versi terbaru)
    \item Koneksi internet stabil
\end{itemize}

%----------------------------------------------------------------------------------------
%   BAB 2: ANTARMUKA PUBLIK (FRONTEND)
%----------------------------------------------------------------------------------------
\chapter{Antarmuka Publik}

Halaman depan website dirancang untuk memberikan informasi lengkap kepada pengunjung dalam satu halaman (Landing Page) yang terdiri dari beberapa segmen yang dapat diakses melalui navigasi halus. Tampilan halaman ini mencerminkan identitas dan profesionalisme PT. Surya Kencana Gemilang Teknik kepada pengunjung.

\section{Struktur Halaman Utama}
Halaman utama menyajikan informasi perusahaan secara komprehensif dalam format satu halaman yang responsif. Setiap bagian memiliki fungsi spesifik dalam menyampaikan informasi perusahaan:

\subsection{Hero Section}
\begin{itemize}
    \item \textbf{Slider Gambar Latar Belakang}: Area banner utama dengan gambar latar belakang yang dapat diubah melalui panel admin.
    \item \textbf{Judul Besar}: Teks besar yang menjadi fokus utama dari pesan utama perusahaan.
    \item \textbf{Lokasi Perusahaan}: Informasi lokasi yang ditampilkan dengan menarik.
    \item \textbf{Statistik Ringkas}: Informasi numerik penting seperti Jumlah Mesin, Klien, Pelanggan, dan Pengalaman perusahaan.
\end{itemize}

% Placeholder untuk screenshot Hero Section
\begin{figure}[H]
    \centering
    \includegraphics[width=1\textwidth]{hero.png}
    \caption{Hero Section - Bagian Atas Halaman Utama}
    \label{fig:frontpage-hero}
\end{figure}

\subsection{Tentang Kami (Visi \& Misi)}
\begin{enumerate}
    \setcounter{enumi}{1} % Mengatur urutan ke 2
    \item Menampilkan visi perusahaan yang menjelaskan tujuan jangka panjang.
    \item Menampilkan daftar poin misi yang dapat diperbarui melalui panel admin.
    \item Bagian ini memberikan wawasan tentang nilai-nilai dan komitmen perusahaan.
\end{enumerate}

% Placeholder untuk screenshot Tentang Kami
\begin{figure}[H]
    \centering
    \includegraphics[width=0.8\textwidth]{about.png}
    \caption{Bagian Tentang Kami (Visi \& Misi)}
    \label{fig:frontpage-about}
\end{figure}

\subsection{Produk}
\begin{enumerate}
    \setcounter{enumi}{2} % Mengatur urutan ke 3
    \item Katalog produk unggulan perusahaan yang ditampilkan dalam bentuk kartu.
    \item Setiap produk menampilkan gambar, deskripsi, dan harga (jika tidak disembunyikan).
    \item Pengunjung dapat melihat detail produk lebih lengkap dengan klik pada kartu produk.
    \item Terdapat tombol langsung ke WhatsApp untuk pemesanan atau informasi lebih lanjut.
\end{enumerate}

% Placeholder untuk screenshot bagian Produk
\begin{figure}[H]
    \centering
    \includegraphics[width=0.8\textwidth]{products.png}
    \caption{Bagian Produk - Katalog Produk Unggulan}
    \label{fig:frontpage-products}
\end{figure}

\subsection{Gallery (Riwayat)}
\begin{enumerate}
    \setcounter{enumi}{3} % Mengatur urutan ke 4
    \item Linimasa sejarah perusahaan yang menampilkan pencapaian penting berdasarkan tahun.
    \item Menampilkan milestone dan perkembangan perusahaan sejak berdiri.
    \item Dapat dikelola melalui panel admin sesuai kebutuhan dokumentasi.
\end{enumerate}

% Placeholder untuk screenshot Gallery Riwayat
\begin{figure}[H]
    \centering
    \includegraphics[width=0.8\textwidth]{gallery.png}
    \caption{Bagian Gallery (Riwayat) - Linimasa Sejarah Perusahaan}
    \label{fig:frontpage-gallery}
\end{figure}

\subsection{Klien (Our Client)}
\begin{enumerate}
    \setcounter{enumi}{4} % Mengatur urutan ke 5
    \item Tampilan grid logo-logo perusahaan yang menjadi mitra kerja.
    \item Menunjukkan kredibilitas dan kepercayaan dari pelanggan serta mitra bisnis.
    \item Logo dapat diupdate melalui panel admin sesuai dengan perkembangan kemitraan.
\end{enumerate}

% Placeholder untuk screenshot Our Client
\begin{figure}[H]
    \centering
    \includegraphics[width=0.8\textwidth]{clients.png}
    \caption{Bagian Our Client - Logo Mitra Kerja}
    \label{fig:frontpage-clients}
\end{figure}

\subsection{Testimoni}
\begin{enumerate}
    \setcounter{enumi}{5} % Mengatur urutan ke 6
    \item Slider ulasan dari pelanggan sebelumnya yang mencerminkan kepuasan layanan.
    \item Menunjukkan kredibilitas dan reputasi perusahaan di mata pelanggan.
    \item Ulasan dapat dikelola dan diperbarui melalui panel admin.
\end{enumerate}

% Placeholder untuk screenshot Testimoni
\begin{figure}[H]
    \centering
    \includegraphics[width=0.8\textwidth]{testimonials.png}
    \caption{Bagian Testimoni - Ulasan Pelanggan}
    \label{fig:frontpage-testimonials}
\end{figure}

\subsection{Kontak}
\begin{enumerate}
    \setcounter{enumi}{6} % Mengatur urutan ke 7
    \item Menyediakan informasi alamat lengkap kantor/pabrik perusahaan.
    \item Menampilkan peta lokasi (Google Maps yang dapat dikonfigurasi).
    \item Menyediakan form kontak WhatsApp untuk komunikasi langsung.
\end{enumerate}

% Placeholder untuk screenshot bagian Kontak
\begin{figure}[H]
    \centering
    \includegraphics[width=0.8\textwidth]{contact.png}
    \caption{Bagian Kontak - Informasi Komunikasi Perusahaan}
    \label{fig:frontpage-contact}
\end{figure}

% Placeholder untuk screenshot tampilan keseluruhan halaman utama
\begin{figure}[H]
    \centering
    \includegraphics[width=0.8\textwidth]{placeholder_frontpage_full.png}
    \caption{Tampilan Keseluruhan Halaman Utama Website PT. Surya Kencana}
    \label{fig:frontpage-full}
\end{figure}

\section{Fitur Interaktif}
Antarmuka publik dilengkapi dengan berbagai fitur interaktif untuk meningkatkan pengalaman pengguna:

\subsection{Floating WhatsApp}
\begin{itemize}
    \item Tombol WhatsApp melayang di pojok kanan bawah untuk komunikasi instan.
    \item Terhubung langsung dengan nomor yang telah dikonfigurasi di panel admin.
    \item Memudahkan pengunjung untuk menghubungi perusahaan secara langsung.
\end{itemize}

% Placeholder untuk screenshot Floating WhatsApp
\begin{figure}[H]
    \centering
    \includegraphics[width=0.8\textwidth]{placeholder_feature_whatsapp.png}
    \caption{Fitur Floating WhatsApp}
    \label{fig:feature-whatsapp}
\end{figure}

\subsection{Detail Produk}
\begin{itemize}
    \item Klik pada kartu produk untuk membuka galeri gambar produk secara penuh (lightbox).
    \item Menampilkan deskripsi lengkap produk beserta informasi kontak.
    \item Memungkinkan pengunjung untuk melihat detail produk secara menyeluruh.
\end{itemize}

% Placeholder untuk screenshot Detail Produk
\begin{figure}[H]
    \centering
    \includegraphics[width=0.8\textwidth]{placeholder_feature_product_detail.png}
    \caption{Fitur Detail Produk}
    \label{fig:feature-product-detail}
\end{figure}

\subsection{Mobile Menu}
\begin{itemize}
    \item Pada tampilan seluler, menu navigasi tersimpan dalam tombol \textit{hamburger} di pojok kanan atas.
    \item Memberikan pengalaman mobile yang optimal dengan navigasi yang mudah diakses.
    \item Memastikan aksesibilitas menu di berbagai ukuran perangkat.
\end{itemize}

% Placeholder untuk screenshot Mobile Menu
\begin{figure}[H]
    \centering
    \includegraphics[width=0.8\textwidth]{placeholder_feature_mobile_menu.png}
    \caption{Fitur Mobile Menu}
    \label{fig:feature-mobile-menu}
\end{figure}

\subsection{Smooth Scrolling}
\begin{itemize}
    \item Navigasi antar bagian halaman menggunakan efek gulir halus untuk pengalaman pengguna yang lebih baik.
    \item Membuat transisi antar bagian terasa lebih profesional dan enak dilihat.
    \item Meningkatkan kenyamanan pengguna saat menjelajahi halaman.
\end{itemize}

% Placeholder untuk screenshot Smooth Scrolling
\begin{figure}[H]
    \centering
    \includegraphics[width=0.8\textwidth]{placeholder_feature_smooth_scroll.png}
    \caption{Fitur Smooth Scrolling}
    \label{fig:feature-smooth-scroll}
\end{figure}

\subsection{Responsive Design}
\begin{itemize}
    \item Desain yang menyesuaikan secara otomatis dengan ukuran layar perangkat pengguna.
    \item Memastikan tampilan optimal di desktop, tablet, maupun ponsel.
    \item Memberikan pengalaman konsisten di berbagai perangkat.
\end{itemize}

% Placeholder untuk screenshot Responsive Design
\begin{figure}[H]
    \centering
    \includegraphics[width=0.8\textwidth]{placeholder_feature_responsive.png}
    \caption{Fitur Responsive Design}
    \label{fig:feature-responsive}
\end{figure}

%----------------------------------------------------------------------------------------
%   BAB 3: LOGIN SISTEM
%----------------------------------------------------------------------------------------
\chapter{Login Administrator}

Untuk masuk ke area pengelolaan konten, ikuti langkah berikut:

\section{Langkah-Langkah Login}
Proses login merupakan langkah penting untuk mengakses sistem administrasi. Pastikan Anda memiliki kredensial yang valid sebelum melanjutkan.

\subsection{Langkah 1: Akses Halaman Login}
\begin{enumerate}
    \item Buka browser web (Chrome, Firefox, Safari, atau Edge) di perangkat Anda.
    \item Ketik alamat website pada address bar: \url{/login} atau \url{/admin}.
    \item Tekan tombol Enter atau klik tombol Go di browser.
\end{enumerate}

% Placeholder untuk screenshot halaman login - tampilan awal
\begin{figure}[H]
    \centering
    \includegraphics[width=0.8\textwidth]{placeholder_login_page_1.png}
    \caption{Halaman Login - Tampilan Awal}
    \label{fig:login-page-1}
\end{figure}

\subsection{Langkah 2: Masukkan Username}
\begin{enumerate}
    \setcounter{enumi}{1} % Mengatur nomor urutan ke 2
    \item Pada kotak pertama yang bertuliskan "Username", masukkan username Anda.
    \item Username biasanya berupa kombinasi huruf dan/atau angka yang telah ditetapkan oleh administrator sistem.
    \item Pastikan tidak ada spasi tambahan di awal atau akhir username.
\end{enumerate}

% Placeholder untuk screenshot form username
\begin{figure}[H]
    \centering
    \includegraphics[width=0.8\textwidth]{placeholder_login_username.png}
    \caption{Form Input Username}
    \label{fig:login-username}
\end{figure}

\subsection{Langkah 3: Masukkan Password}
\begin{enumerate}
    \setcounter{enumi}{2} % Mengatur nomor urutan ke 3
    \item Pada kotak kedua yang bertuliskan "Password", masukkan password yang sesuai dengan username Anda.
    \item Pastikan password diketik dengan benar, karena sistem membedakan huruf besar dan kecil (case-sensitive).
    \item Jika ingin melihat karakter yang diketik, klik ikon \textbf{Mata} di sebelah kanan kolom password.
    \item Gunakan tombol Backspace jika perlu mengoreksi karakter yang salah.
\end{enumerate}

% Placeholder untuk screenshot form password
\begin{figure}[H]
    \centering
    \includegraphics[width=0.8\textwidth]{placeholder_login_password.png}
    \caption{Form Input Password dengan Ikon Mata}
    \label{fig:login-password}
\end{figure}

\subsection{Langkah 4: Klik Tombol Login}
\begin{enumerate}
    \setcounter{enumi}{3} % Mengatur nomor urutan ke 4
    \item Setelah memasukkan username dan password dengan benar, klik tombol \textbf{Login} yang berwarna biru dan terletak di bagian bawah form.
    \item Tunggu beberapa saat (1-3 detik) hingga sistem memproses permintaan login.
\end{enumerate}

% Placeholder untuk screenshot tombol login
\begin{figure}[H]
    \centering
    \includegraphics[width=0.8\textwidth]{placeholder_login_button.png}
    \caption{Tombol Login pada Form}
    \label{fig:login-button}
\end{figure}

\subsection{Langkah 5: Verifikasi Login dan Akses Dashboard}
\begin{enumerate}
    \setcounter{enumi}{4} % Mengatur nomor urutan ke 5
    \item Jika login berhasil, Anda akan diarahkan secara otomatis ke halaman \textbf{Dashboard}.
    \item Periksa bagian atas layar untuk memastikan bahwa Anda telah login dengan akun yang benar.
    \item Jika login gagal, periksa kembali username dan password Anda, serta pastikan tidak ada kesalahan pengetikan.
\end{enumerate}

% Placeholder untuk screenshot hasil login sukses ke dashboard
\begin{figure}[H]
    \centering
    \includegraphics[width=0.8\textwidth]{placeholder_login_success.png}
    \caption{Login Berhasil - Akses ke Dashboard}
    \label{fig:login-success}
\end{figure}

\section{Keamanan Akun}
\begin{itemize}
    \item Pastikan untuk menggunakan kombinasi username dan password yang kuat.
    \item Jangan pernah membagikan akun Anda kepada pihak yang tidak berwenang.
    \item Logout setelah selesai menggunakan panel admin, terutama saat menggunakan perangkat bersama.
    \item Ganti password secara berkala untuk menjaga keamanan sistem.
    \item Gunakan fitur "Remember Me" hanya di perangkat pribadi, hindari di perangkat umum.
\end{itemize}

\begin{warningbox}[Keamanan]
Jika gagal login, pesan error akan muncul di bagian atas form. Periksa kembali huruf besar/kecil (case-sensitive). Sistem juga memiliki fitur pembatasan login otomatis setelah beberapa percobaan gagal untuk mencegah serangan brute-force.
\end{warningbox}

%----------------------------------------------------------------------------------------
%   BAB 4: DASHBOARD
%----------------------------------------------------------------------------------------
\chapter{Dashboard}

Dashboard adalah halaman pertama yang muncul setelah login. Halaman ini memberikan ringkasan performa data website dan akses cepat ke fungsi administrasi utama. Dashboard dirancang untuk memberikan informasi penting secara visual dan memudahkan administrator dalam mengakses fitur-fitur penting.

\section{Statistik Utama}
Di bagian atas dashboard, terdapat 4 kartu statistik yang memberikan gambaran cepat tentang kondisi konten website:

\subsection{Total Produk}
\begin{itemize}
    \item Menampilkan jumlah produk aktif di katalog yang ditampilkan di halaman publik.
    \item Data ini mencakup semua produk yang sedang dipromosikan di situs.
    \item Angka ini terhubung langsung dengan modul Manajemen Produk.
\end{itemize}

% Placeholder untuk screenshot kartu Total Produk
\begin{figure}[H]
    \centering
    \includegraphics[width=0.8\textwidth]{placeholder_dashboard_products.png}
    \caption{Kartu Statistik - Total Produk}
    \label{fig:dashboard-products}
\end{figure}

\subsection{Total Klien}
\begin{itemize}
    \item Menunjukkan jumlah mitra kerja yang ditampilkan di bagian Our Client di situs publik.
    \item Data ini mencerminkan jumlah logo klien yang sedang aktif ditampilkan.
    \item Terhubung langsung dengan modul Manajemen Klien.
\end{itemize}

% Placeholder untuk screenshot kartu Total Klien
\begin{figure}[H]
    \centering
    \includegraphics[width=0.8\textwidth]{placeholder_dashboard_clients.png}
    \caption{Kartu Statistik - Total Klien}
    \label{fig:dashboard-clients}
\end{figure}

\subsection{Total Testimoni}
\begin{itemize}
    \item Menampilkan jumlah ulasan pelanggan yang ditampilkan di halaman publik.
    \item Data ini mencakup semua testimoni yang sedang tayang di situs.
    \item Terhubung langsung dengan modul Manajemen Testimoni.
\end{itemize}

% Placeholder untuk screenshot kartu Total Testimoni
\begin{figure}[H]
    \centering
    \includegraphics[width=0.8\textwidth]{placeholder_dashboard_testimonials.png}
    \caption{Kartu Statistik - Total Testimoni}
    \label{fig:dashboard-testimonials}
\end{figure}

\subsection{Total Admin}
\begin{itemize}
    \item Menunjukkan jumlah akun pengelola sistem termasuk super admin.
    \item Data ini mencakup semua akun yang memiliki akses ke sistem administrasi.
    \item Terhubung langsung dengan modul Manajemen Admin (untuk super admin).
\end{itemize}

% Placeholder untuk screenshot kartu Total Admin
\begin{figure}[H]
    \centering
    \includegraphics[width=0.8\textwidth]{placeholder_dashboard_admins.png}
    \caption{Kartu Statistik - Total Admin}
    \label{fig:dashboard-admins}
\end{figure}

\section{Aktivitas Terbaru}
Bagian ini menampilkan dua panel daftar yang menunjukkan konten terbaru yang ditambahkan ke sistem:

\subsection{Panel Produk Terbaru}
\begin{enumerate}
    \item Menampilkan daftar 5 produk terakhir yang ditambahkan ke sistem.
    \item Setiap entri mencakup nama produk dan harga (jika ditampilkan).
    \item Klik tombol \menu{Lihat Semua} di header panel untuk menuju ke halaman \textbf{Manajemen Produk}.
\end{enumerate}

% Placeholder untuk screenshot panel Produk Terbaru
\begin{figure}[H]
    \centering
    \includegraphics[width=0.8\textwidth]{placeholder_dashboard_recent_products.png}
    \caption{Panel Aktivitas Terbaru - Produk}
    \label{fig:dashboard-recent-products}
\end{figure}

\subsection{Panel Testimoni Terbaru}
\begin{enumerate}
    \item Menampilkan daftar testimoni pelanggan yang baru-baru ini ditambahkan.
    \item Setiap entri mencakup nama pengirim dan tanggal penambahan.
    \item Klik tombol \menu{Lihat Semua} di header panel untuk menuju ke halaman \textbf{Manajemen Testimoni}.
\end{enumerate}

% Placeholder untuk screenshot panel Testimoni Terbaru
\begin{figure}[H]
    \centering
    \includegraphics[width=0.8\textwidth]{placeholder_dashboard_recent_testimonials.png}
    \caption{Panel Aktivitas Terbaru - Testimoni}
    \label{fig:dashboard-recent-testimonials}
\end{figure}

\section{Navigasi Cepat}
Dashboard juga menyediakan tombol akses cepat ke modul utama sistem:

\begin{itemize}
    \item Tombol-tombol ini terletak di bagian tengah dashboard.
    \item Memungkinkan administrator untuk langsung menuju ke fitur yang paling sering digunakan.
    \item Tautan langsung ke Produk, Klien, Testimoni, Hero, dan halaman penting lainnya.
\end{itemize}

% Placeholder untuk screenshot bagian navigasi cepat
\begin{figure}[H]
    \centering
    \includegraphics[width=0.8\textwidth]{placeholder_dashboard_quick_nav.png}
    \caption{Bagian Navigasi Cepat di Dashboard}
    \label{fig:dashboard-quick-nav}
\end{figure}

Klik tombol \menu{Lihat Semua} pada header panel untuk menuju ke halaman manajemen masing-masing. Panel ini memudahkan administrator untuk memantau konten terkini yang baru saja ditambahkan ke sistem.

%----------------------------------------------------------------------------------------
%   BAB 5: MANAJEMEN HERO & VISUAL
%----------------------------------------------------------------------------------------
\chapter{Manajemen Hero (Banner Utama)}

Menu ini digunakan untuk mengatur tampilan paling atas website (Banner). Bagian Hero adalah elemen pertama yang dilihat pengunjung dan berfungsi sebagai representasi visual utama dari perusahaan.

\section{Pengaturan Gambar Slider}
Anda dapat mengupload lebih dari satu gambar background yang akan berganti secara otomatis (slider) untuk memberikan kesan dinamis pada tampilan utama website.

\subsection{Langkah 1: Akses Menu Hero}
\begin{enumerate}
    \item Login ke panel administrasi dengan akun yang memiliki akses.
    \item Klik menu \menu{Hero} di sidebar navigasi sebelah kiri.
    \item Tunggu beberapa detik hingga halaman Hero Management dimuat.
\end{enumerate}

% Placeholder untuk screenshot akses menu hero
\begin{figure}[H]
    \centering
    \includegraphics[width=0.8\textwidth]{placeholder_hero_access.png}
    \caption{Akses Menu Hero dari Sidebar}
    \label{fig:hero-access}
\end{figure}

\subsection{Langkah 2: Upload Gambar Background}
\begin{enumerate}
    \setcounter{enumi}{1} % Mengatur urutan ke 2
    \item Di bagian \textbf{Gambar Background Slider}, temukan tombol atau area upload bertuliskan \textbf{"Klik untuk upload gambar"}.
    \item Klik area tersebut untuk membuka dialog pemilihan file dari komputer Anda.
    \item Pilih satu atau beberapa gambar sekaligus dari komputer Anda (format yang didukung: JPEG, PNG, WebP).
    \item Sesuaikan resolusi gambar untuk tampilan terbaik (disarankan 1920x1080 piksel).
\end{enumerate}

% Placeholder untuk screenshot area upload gambar
\begin{figure}[H]
    \centering
    \includegraphics[width=0.8\textwidth]{placeholder_hero_upload_area.png}
    \caption{Area Upload Gambar Background}
    \label{fig:hero-upload-area}
\end{figure}

\subsection{Langkah 3: Pengelolaan Gambar}
\begin{enumerate}
    \setcounter{enumi}{2} % Mengatur urutan ke 3
    \item Gambar baru akan muncul di daftar preview dan akan ditampilkan sebagai bagian dari slider.
    \item Untuk menghapus gambar lama, arahkan kursor ke gambar tersebut dan klik ikon \textbf{Silang (X)} merah di pojok kanan atas gambar.
    \item Untuk mengganti gambar yang sudah ada, Anda bisa meng-upload gambar baru dan menghapus gambar lama.
    \item Urutan gambar pada slider akan mengikuti urutan gambar dalam daftar preview.
\end{enumerate}

% Placeholder untuk screenshot pengelolaan gambar
\begin{figure}[H]
    \centering
    \includegraphics[width=0.8\textwidth]{placeholder_hero_manage_images.png}
    \caption{Pengelolaan Gambar Slider di Hero Section}
    \label{fig:hero-manage-images}
\end{figure}

\subsection{Langkah 4: Penyesuaian Urutan Gambar}
\begin{enumerate}
    \setcounter{enumi}{3} % Mengatur urutan ke 4
    \item Anda dapat menyesuaikan urutan gambar dengan cara drag-and-drop gambar ke posisi yang diinginkan.
    \item Gambar pertama dalam daftar akan menjadi gambar utama yang muncul pertama kali di slider.
    \item Atur urutan gambar berdasarkan prioritas atau kronologis sesuai kebutuhan branding.
\end{enumerate}

% Placeholder untuk screenshot penyesuaian urutan
\begin{figure}[H]
    \centering
    \includegraphics[width=0.8\textwidth]{placeholder_hero_order_images.png}
    \caption{Penyesuaian Urutan Gambar Slider}
    \label{fig:hero-order-images}
\end{figure}

\section{Pengaturan Teks \& Statistik}
Di bawah bagian gambar, terdapat form untuk mengubah teks yang tampil di banner, termasuk elemen yang menarik perhatian pengunjung:

\subsection{Pengaturan Lokasi}
\begin{itemize}
    \item Pada kolom \textbf{Lokasi}, masukkan teks lokasi perusahaan yang ditampilkan secara menarik (misal: "Jakarta, Indonesia").
    \item Teks ini akan muncul di bawah judul besar dan memberikan informasi geografis tentang perusahaan.
    \item Pastikan penulisan lokasi konsisten dengan format yang Anda gunakan di sumber lain.
\end{itemize}

% Placeholder untuk screenshot pengaturan lokasi
\begin{figure}[H]
    \centering
    \includegraphics[width=0.8\textwidth]{placeholder_hero_location.png}
    \caption{Pengaturan Teks Lokasi di Hero Section}
    \label{fig:hero-location}
\end{figure}

\subsection{Pengaturan Judul Utama}
\begin{itemize}
    \item Pada kolom \textbf{Judul Utama}, masukkan headline besar yang menjadi fokus utama (misal: "Mesin Custom untuk Industri").
    \item Judul ini adalah elemen teks paling menonjol di bagian Hero.
    \item Buat judul yang singkat namun menarik dan mencerminkan nilai utama perusahaan.
\end{itemize}

% Placeholder untuk screenshot pengaturan judul utama
\begin{figure}[H]
    \centering
    \includegraphics[width=0.8\textwidth]{placeholder_hero_main_title.png}
    \caption{Pengaturan Judul Utama di Hero Section}
    \label{fig:hero-main-title}
\end{figure}

\subsection{Pengaturan Subjudul}
\begin{itemize}
    \item Kolom \textbf{Subjudul} digunakan untuk teks tambahan yang memberikan penjelasan lebih lanjut.
    \item Gunakan subjudul untuk memperkuat pesan dari judul utama.
    \item Subjudul biasanya tampil dalam ukuran yang lebih kecil dari judul utama.
\end{itemize}

% Placeholder untuk screenshot pengaturan subjudul
\begin{figure}[H]
    \centering
    \includegraphics[width=0.8\textwidth]{placeholder_hero_subtitle.png}
    \caption{Pengaturan Subjudul di Hero Section}
    \label{fig:hero-subtitle}
\end{figure}

\subsection{Pengaturan Statistik}
\begin{itemize}
    \item Pada bagian \textbf{Statistik}, masukkan angka untuk data statistik perusahaan yang ditampilkan dalam bentuk visual menarik:
    \item \textbf{Jumlah Mesin}: Jumlah total mesin yang pernah dibuat/diproduksi
    \item \textbf{Jumlah Klien}: Jumlah klien yang pernah menggunakan jasa/produk
    \item \textbf{Jumlah Pelanggan}: Jumlah pelanggan yang puas
    \item \textbf{Tahun Pengalaman}: Tahun berdirinya perusahaan atau pengalaman di industri
    \item \textbf{Tahun Kepercayaan}: Tahun kepercayaan pelanggan terhadap perusahaan
\end{itemize}

% Placeholder untuk screenshot pengaturan statistik
\begin{figure}[H]
    \centering
    \includegraphics[width=0.8\textwidth]{placeholder_hero_statistics.png}
    \caption{Pengaturan Statistik di Hero Section}
    \label{fig:hero-statistics}
\end{figure}

\section{Pratinjau Perubahan}
Sebelum menyimpan, Anda dapat melihat pratinjau perubahan yang telah Anda buat:

\begin{itemize}
    \item Klik bagian pratinjau untuk melihat bagaimana tampilan akan terlihat di halaman publik.
    \item Periksa apakah semua elemen tampil dengan benar dan sesuai harapan.
    \item Bandingkan dengan tampilan sebelumnya untuk memastikan perubahan sesuai.
\end{itemize}

% Placeholder untuk screenshot pratinjau
\begin{figure}[H]
    \centering
    \includegraphics[width=0.8\textwidth]{placeholder_hero_preview.png}
    \caption{Pratinjau Perubahan di Hero Section}
    \label{fig:hero-preview}
\end{figure}

\section{Tips Optimasi Hero Section}
\begin{itemize}
    \item Gunakan gambar beresolusi tinggi (minimal 1920x1080 piksel) untuk kualitas tampilan terbaik
    \item Pastikan ukuran file gambar tidak terlalu besar untuk menjaga kecepatan loading halaman
    \item Gunakan format WebP jika memungkinkan untuk ukuran file yang lebih kecil
    \item Teks pada Hero sebaiknya singkat, jelas, dan mencerminkan nilai utama perusahaan
    \item Pilih gambar yang secara visual mewakili layanan atau produk utama perusahaan
    \item Pastikan teks tetap terbaca dengan jelas di atas gambar latar belakang
\end{itemize}

Klik tombol \menu{Simpan Perubahan} di bagian bawah untuk menerapkan semua update (gambar dan teks).

%----------------------------------------------------------------------------------------
%   BAB 6: MANAJEMEN PRODUK
%----------------------------------------------------------------------------------------
\chapter{Manajemen Produk}

Menu ini adalah inti dari website, tempat Anda mengelola katalog mesin dan informasi produk secara lengkap. Produk yang dikelola di sini akan langsung tampil di halaman publik untuk dilihat oleh pengunjung website.

\section{Langkah 1: Akses Menu Produk}
\begin{enumerate}
    \item Setelah login ke panel administrasi, temukan dan klik menu \menu{Produk} di sidebar navigasi sebelah kiri.
    \item Tunggu hingga halaman daftar produk dimuat sepenuhnya. Anda akan melihat daftar produk yang telah dibuat sebelumnya.
\end{enumerate}

% Placeholder untuk screenshot akses menu produk
\begin{figure}[H]
    \centering
    \includegraphics[width=0.8\textwidth]{placeholder_product_menu_access.png}
    \caption{Akses Menu Produk dari Sidebar Administrasi}
    \label{fig:product-menu-access}
\end{figure}

\section{Langkah 2: Menambah Produk Baru}
\begin{enumerate}
    \setcounter{enumi}{1} % Atur urutan ke 2
    \item Di pojok kanan atas halaman daftar produk, klik tombol hijau bertuliskan \menu{+ Tambah Produk}.
    \item Form penambahan produk akan muncul dalam modal atau halaman baru.
\end{enumerate}

\subsection{Isian Form Penambahan Produk}
Saat menambahkan produk baru, Anda perlu mengisi beberapa informasi penting:

\subsubsection{Nama Produk}
\begin{itemize}
    \item Wajib diisi dengan nama produk yang jelas dan deskriptif (contoh: "Mesin CNC 5 Axis").
    \item Gunakan penamaan yang konsisten dengan standar penamaan produk perusahaan.
    \item Hindari penggunaan karakter khusus yang berlebihan.
\end{itemize}

% Placeholder untuk gambar form nama produk
\begin{figure}[H]
    \centering
    \includegraphics[width=0.8\textwidth]{placeholder_product_name.png}
    \caption{Input Nama Produk}
    \label{fig:product-name}
\end{figure}

\subsubsection{Harga Produk}
\begin{itemize}
    \item Masukkan angka harga tanpa simbol mata uang, hanya angka (contoh: 250000000).
    \item Sistem akan otomatis memformat tampilan harga menjadi format mata uang Indonesia (Rp) dengan pemisah ribuan.
    \item Harga hanya akan muncul di halaman publik jika opsi "Sembunyikan Harga" tidak dicentang.
\end{itemize}

% Placeholder untuk gambar form harga produk
\begin{figure}[H]
    \centering
    \includegraphics[width=0.8\textwidth]{placeholder_product_price.png}
    \caption{Input Harga Produk}
    \label{fig:product-price}
\end{figure}

\subsubsection{Opsi Sembunyikan Harga}
\begin{itemize}
    \item Centang kotak ini jika Anda tidak ingin harga tampil di publik.
    \item Jika dicentang, harga akan diganti dengan teks "Hubungi Kami" untuk mendorong interaksi langsung dengan pelanggan.
    \item Opsi ini berguna untuk produk yang harganya perlu dinegosiasikan atau informasi harganya sensitif.
\end{itemize}

% Placeholder untuk gambar opsi sembunyikan harga
\begin{figure}[H]
    \centering
    \includegraphics[width=0.8\textwidth]{placeholder_hide_price_option.png}
    \caption{Opsi Sembunyikan Harga Produk}
    \label{fig:hide-price-option}
\end{figure}

\subsubsection{Deskripsi Produk}
\begin{itemize}
    \item Teks penjelasan detail mengenai spesifikasi, keunggulan, dan fitur produk.
    \item Gunakan format paragraf yang mudah dipahami oleh pelanggan potensial.
    \item Cantumkan informasi penting seperti spesifikasi teknis, material, ukuran, dan keunggulan kompetitif.
\end{itemize}

% Placeholder untuk gambar form deskripsi produk
\begin{figure}[H]
    \centering
    \includegraphics[width=0.8\textwidth]{placeholder_product_description.png}
    \caption{Form Deskripsi Produk}
    \label{fig:product-description}
\end{figure}

\subsubsection{Gambar Produk Utama}
\begin{itemize}
    \item Gambar wajib diunggah (format: JPEG, PNG, WebP) yang akan tampil di kartu produk di halaman publik.
    \item Gunakan gambar beresolusi tinggi untuk tampilan profesional.
    \item Disarankan gambar memiliki rasio tertentu (misalnya 4:3 atau 1:1) untuk konsistensi tampilan.
\end{itemize}

% Placeholder untuk area upload gambar produk utama
\begin{figure}[H]
    \centering
    \includegraphics[width=0.8\textwidth]{placeholder_main_product_image.png}
    \caption{Area Upload Gambar Produk Utama}
    \label{fig:main-product-image}
\end{figure}

\subsubsection{Gambar Album Produk}
\begin{itemize}
    \item Gambar opsional yang akan ditampilkan di galeri detail produk saat pengunjung mengklik produk.
    \item Anda bisa meng-upload lebih dari satu gambar sekaligus untuk menunjukkan detail atau sisi berbeda dari produk.
    \item Gunakan tombol upload ganda untuk memilih lebih dari satu file gambar sekaligus.
\end{itemize}

% Placeholder untuk area upload album produk
\begin{figure}[H]
    \centering
    \includegraphics[width=0.8\textwidth]{placeholder_product_album.png}
    \caption{Area Upload Album Gambar Produk}
    \label{fig:product-album}
\end{figure}

\subsection{Menyimpan Produk Baru}
\begin{enumerate}
    \setcounter{enumi}{2} % Atur urutan ke 3
    \item Setelah mengisi semua informasi produk yang diperlukan, klik tombol \menu{Tambah Produk} di bagian bawah form.
    \item Tunggu proses penyimpanan. Setelah berhasil, Anda akan melihat notifikasi sukses dan produk baru muncul di daftar produk.
\end{enumerate}

% Placeholder untuk screenshot form penambahan produk lengkap
\begin{figure}[H]
    \centering
    \includegraphics[width=0.8\textwidth]{placeholder_add_product_full.png}
    \caption{Form Lengkap Penambahan Produk Baru}
    \label{fig:add-product-full}
\end{figure}

\section{Mengelola Produk Eksisting}
Setelah produk ditambahkan, Anda bisa melakukan berbagai operasi pengelolaan:

\subsection{Melihat Daftar Produk}
\begin{itemize}
    \item Halaman daftar produk menampilkan semua produk yang telah dibuat dalam bentuk kartu-kartu.
    \item Setiap kartu menampilkan gambar produk, nama produk, harga (jika tidak disembunyikan), dan tombol aksi.
    \item Gunakan fitur pencarian dan filter jika Anda memiliki banyak produk untuk menemukan produk tertentu.
\end{itemize}

% Placeholder untuk screenshot daftar produk
\begin{figure}[H]
    \centering
    \includegraphics[width=0.8\textwidth]{placeholder_product_list_view.png}
    \caption{Tampilan Daftar Seluruh Produk}
    \label{fig:product-list-view}
\end{figure}

\subsection{Mengatur Urutan Produk (Drag \& Drop)}
Fitur unggulan sistem ini adalah kemampuan mengurutkan produk secara visual untuk menentukan prioritas tampilan di halaman publik:
\begin{enumerate}
    \item Pada halaman daftar produk, klik dan tahan pada kartu produk yang ingin dipindahkan.
    \item Geser (\textit{drag}) ke posisi urutan yang diinginkan.
    \item Lepaskan (\textit{drop}).
    \item Sistem akan otomatis menyimpan urutan baru dan menampilkan notifikasi sukses.
    \item Urutan produk akan langsung terlihat di halaman publik sesuai dengan urutan yang telah ditentukan.
\end{enumerate}

% Placeholder untuk screenshot fitur drag & drop
\begin{figure}[H]
    \centering
    \includegraphics[width=0.8\textwidth]{placeholder_drag_drop_order.png}
    \caption{Fitur Drag \& Drop untuk Mengatur Urutan Produk}
    \label{fig:drag-drop-order}
\end{figure}

\subsection{Mengedit Produk}
\begin{enumerate}
    \item Temukan produk yang ingin diedit di daftar produk.
    \item Klik ikon \textbf{Pensil} (Edit) pada kartu produk.
    \item Form edit akan muncul dengan informasi produk yang sudah terisi.
    \item Lakukan perubahan yang diperlukan (nama, harga, deskripsi, gambar, dll).
    \item Tambahkan atau hapus gambar album produk sesuai kebutuhan.
    \item Klik \menu{Simpan Perubahan} untuk menyimpan update produk.
\end{enumerate}

% Placeholder untuk screenshot proses edit produk
\begin{figure}[H]
    \centering
    \includegraphics[width=0.8\textwidth]{placeholder_edit_product_process.png}
    \caption{Proses Mengedit Informasi Produk}
    \label{fig:edit-product-process}
\end{figure}

\subsection{Menghapus Produk}
\begin{enumerate}
    \item Temukan produk yang ingin dihapus di daftar produk.
    \item Klik ikon \textbf{Sampah} (Delete) pada kartu produk.
    \item Konfirmasi penghapusan pada dialog yang muncul.
    \item Produk yang dihapus tidak akan lagi tampil di halaman publik.
    \item Peringatan: Penghapusan produk tidak dapat dibatalkan (tidak dapat dikembalikan).
\end{enumerate}

% Placeholder untuk screenshot konfirmasi penghapusan produk
\begin{figure}[H]
    \centering
    \includegraphics[width=0.8\textwidth]{placeholder_delete_product_confirm.png}
    \caption{Konfirmasi Penghapusan Produk}
    \label{fig:delete-product-confirm}
\end{figure}

\section{Tips Manajemen Produk}
\begin{itemize}
    \item Gunakan nama produk yang deskriptif dan konsisten untuk kemudahan pencarian dan pengelolaan
    \item Sertakan gambar produk beresolusi tinggi (minimal 800x600 piksel) untuk tampilan profesional
    \item Update harga secara berkala agar tetap akurat dan mencerminkan kondisi pasar
    \item Deskripsikan produk dengan jelas dan lengkap, cantumkan spesifikasi teknis yang relevan
    \item Gunakan fitur "Sembunyikan Harga" untuk produk yang perlu negosiasi langsung atau harga sensitif
    \item Atur urutan produk secara strategis, tempatkan produk unggulan di posisi awal
    \item Gunakan gambar dengan cahaya yang baik dan latar belakang sederhana untuk tampilan profesional
    \item Pastikan deskripsi produk mencakup keunggulan kompetitif dan manfaat bagi pelanggan
\end{itemize}

Setelah menambahkan atau mengedit produk, informasi tersebut akan langsung terlihat di halaman publik website sesuai dengan pengaturan yang telah Anda tentukan.

%----------------------------------------------------------------------------------------
%   BAB 7: MANAJEMEN KLIEN & TESTIMONI
%----------------------------------------------------------------------------------------
\chapter{Manajemen Klien \& Testimoni}

\section{Manajemen Klien (Our Client)}
Menu ini untuk menampilkan logo mitra kerja sebagai bentuk kepercayaan dan hubungan kerja dengan perusahaan lain. Bagian ini penting dalam membangun kredibilitas perusahaan.

\begin{enumerate}
    \item Klik menu \menu{Our Client}.
    \item Klik tombol \menu{+ Tambah Client} di pojok kanan atas.
    \item Isi formulir yang muncul:
        \begin{itemize}
            \item \textbf{Nama Client}: Masukkan nama resmi perusahaan klien/mitra.
            \item \textbf{Upload Logo}: Upload logo resmi perusahaan klien. \textit{Saran: Gunakan file PNG dengan latar transparan agar tampil rapi, atau file JPG/WEBP dengan latar putih.}
        \end{itemize}
    \item Klik tombol \menu{Simpan} untuk menyimpan data klien baru.
    \item Anda juga bisa melakukan \textit{Drag \& Drop} untuk mengatur urutan logo klien berdasarkan prioritas atau abjad.
\end{enumerate}

% Placeholder untuk screenshot form manajemen klien
\begin{figure}[H]
    \centering
    \includegraphics[width=0.8\textwidth]{placeholder_client_management.png}
    \caption{Form Manajemen Klien (Our Client)}
    \label{fig:client-management}
\end{figure}

\section{Manajemen Testimoni}
Menu ini untuk menampilkan ulasan pelanggan sebagai bentuk kepuasan dan rekomendasi terhadap layanan/perusahaan. Testimoni berperan penting dalam membangun kepercayaan calon klien.

\begin{enumerate}
    \item Klik menu \menu{Testimoni}.
    \item Klik tombol \menu{+ Tambah Testimoni} di pojok kanan atas.
    \item Isi formulir yang muncul secara lengkap:
        \begin{itemize}
            \item \textbf{Nama Klien}: Nama orang yang memberi ulasan (kontak utama atau perwakilan klien).
            \item \textbf{Institusi}: Jabatan atau nama perusahaan asal klien yang memberi testimoni.
            \item \textbf{Testimoni}: Isi teks ulasan yang asli dan otentik. Pastikan testimonial mencerminkan pengalaman nyata.
            \item \textbf{Tanggal}: Tanggal testimoni diberikan (berpengaruh pada urutan tampilan tanggal).
        \end{itemize}
    \item Klik tombol \menu{Simpan} untuk menyimpan testimoni baru.
    \item Fitur \textit{Drag \& Drop} juga tersedia untuk mengatur urutan tampilan testimoni di slider berdasarkan prioritas atau relevansi.
\end{enumerate}

% Placeholder untuk screenshot form manajemen testimoni
\begin{figure}[H]
    \centering
    \includegraphics[width=0.8\textwidth]{placeholder_testimonial_management.png}
    \caption{Form Manajemen Testimoni}
    \label{fig:testimonial-management}
\end{figure}

\section{Tips Manajemen Klien \& Testimoni}
\begin{itemize}
    \item Pastikan logo klien memiliki resolusi tinggi agar tampil jelas di halaman publik
    \item Gunakan logo dengan latar transparan (format PNG) untuk hasil terbaik
    \item Urutkan klien berdasarkan tingkat kepentingan atau abjad untuk tampilan profesional
    \item Pastikan testimoni berasal dari klien sesungguhnya dan telah diotorisasi
    \item Gunakan testimonial yang spesifik dan mencantumkan hasil nyata dari kerjasama
    \item Perbarui testimonial secara berkala untuk menjaga relevansi
\end{itemize}

%----------------------------------------------------------------------------------------
%   BAB 8: MANAJEMEN HALAMAN LAINNYA
%----------------------------------------------------------------------------------------
\chapter{Manajemen Halaman Lainnya}

\section{Visi \& Misi}
Bagian ini menampilkan filosofi dan tujuan strategis perusahaan. Visi dan Misi merupakan elemen penting dalam menyampaikan identitas dan arah perusahaan kepada pengunjung.

\begin{enumerate}
    \item Masuk ke menu \menu{Visi \& Misi}.
    \item \textbf{Visi}: Edit teks pada kotak Visi. Teks visi sebaiknya mencerminkan cita-cita jangka panjang perusahaan.
    \item \textbf{Misi}:
        \begin{itemize}
            \item Edit poin misi yang sudah ada langsung pada kolom teks. Setiap poin misi sebaiknya spesifik dan menggambarkan langkah-langkah untuk mencapai visi.
            \item Klik ikon \textbf{Sampah} di sebelah kanan poin untuk menghapus poin misi yang tidak relevan lagi.
            \item Klik tombol \menu{+ Tambah Poin Misi} di bagian bawah untuk menambah baris baru jika perusahaan memiliki lebih banyak poin misi.
        \end{itemize}
    \item Panel kanan (desktop) atau bawah (mobile) akan menampilkan \textbf{Pratinjau (Preview)} secara \textit{real-time}, sehingga Anda dapat langsung melihat bagaimana tampilan akan terlihat di halaman publik.
    \item Klik tombol \menu{Simpan Perubahan} di bagian bawah setelah selesai melakukan perubahan.
\end{enumerate}

% Placeholder untuk screenshot form visi & misi
\begin{figure}[H]
    \centering
    \includegraphics[width=0.8\textwidth]{placeholder_vision_mission.png}
    \caption{Form Manajemen Visi \& Misi}
    \label{fig:vision-mission}
\end{figure}

\section{Kontak}
Bagian ini mengelola informasi kontak yang tampil di Footer dan halaman Kontak. Ini adalah titik penting bagi pengunjung untuk menghubungi perusahaan.

\begin{itemize}
    \item \textbf{Email}: Alamat email resmi perusahaan yang dapat dihubungi oleh klien/pelanggan.
    \item \textbf{Nomor WhatsApp}: Masukkan nomor WhatsApp aktif (boleh format 08xx atau +62xx). Sistem otomatis memformatnya menjadi link WhatsApp yang dapat diklik langsung oleh pengunjung.
    \item \textbf{Alamat}: Alamat lengkap kantor/pabrik perusahaan yang juga ditampilkan di peta.
    \item \textbf{Nama Lokasi Maps}: Nama tempat untuk pencarian peta Google yang akan membantu pengunjung dalam menemukan lokasi.
    \item \textbf{Tautan Maps (Opsional)}: Link embed atau link share dari Google Maps agar peta lebih akurat dan interaktif di halaman publik.
\end{itemize}

Hasil inputan akan langsung terlihat di panel \textbf{Pratinjau} di sebelah kanan, sehingga Anda dapat memastikan tampilan dan fungsi bekerja dengan benar.

% Placeholder untuk screenshot form kontak
\begin{figure}[H]
    \centering
    \includegraphics[width=0.8\textwidth]{placeholder_contact_form.png}
    \caption{Form Manajemen Kontak}
    \label{fig:contact-form}
\end{figure}

\section{Riwayat Perusahaan (Company History)}
Bagian ini menampilkan \textit{milestone} atau sejarah penting dan pencapaian perusahaan secara kronologis, yang penting untuk membangun kredibilitas dan kepercayaan.

\begin{enumerate}
    \item Klik menu \menu{Riwayat Perusahaan}.
    \item Tambah riwayat baru dengan mengisi informasi secara lengkap:
        \begin{itemize}
            \item \textbf{Tahun}: Tahun terjadinya milestone penting.
            \item \textbf{Judul Peristiwa}: Deskripsi singkat dari peristiwa penting.
            \item \textbf{Deskripsi}: Penjelasan lengkap tentang peristiwa dan dampaknya terhadap perusahaan.
            \item \textbf{Gambar Dokumentasi}: (Opsional) Gambar yang merepresentasikan peristiwa tersebut.
        \end{itemize}
    \item Gunakan fitur \textit{Drag \& Drop} untuk mengatur urutan sejarah berdasarkan kronologi (biasanya dari tahun terlama ke terbaru atau sebaliknya sesuai kebutuhan naratif).
\end{enumerate}

% Placeholder untuk screenshot form riwayat perusahaan
\begin{figure}[H]
    \centering
    \includegraphics[width=0.8\textwidth]{placeholder_company_history.png}
    \caption{Form Manajemen Riwayat Perusahaan}
    \label{fig:company-history}
\end{figure}

\section{Tips Manajemen Halaman Lainnya}
\begin{itemize}
    \item Pastikan Visi dan Misi tetap relevan dengan arah pengembangan perusahaan
    \item Gunakan bahasa yang jelas dan profesional dalam semua bagian
    \item Update informasi kontak secara berkala untuk memastikan aksesibilitas
    \item Sertakan gambar dokumentasi berkualitas tinggi untuk riwayat perusahaan
    \item Atur urutan riwayat secara kronologis untuk narasi yang mudah dipahami
    \item Gunakan nomor WhatsApp aktif dan termonitor untuk menjaga komunikasi dengan pelanggan
\end{itemize}

%----------------------------------------------------------------------------------------
%   BAB 9: PENGATURAN SITUS
%----------------------------------------------------------------------------------------
\chapter{Pengaturan Situs (Global)}

Menu \textbf{Pengaturan Situs} memberikan kendali penuh atas elemen-elemen tampilan global website tanpa perlu mengubah kode program. Pengaturan ini memungkinkan penyesuaian tampilan secara menyeluruh untuk semua bagian website.

\section{Halaman Pengaturan}
Form pengaturan dibagi menjadi 3 halaman (tab) untuk memudahkan navigasi dan pengelompokan fungsi:

\subsection{Halaman 1: Info Dasar \& Hero}
Tab ini berisi pengaturan informasi dasar perusahaan dan elemen-elemen penting di bagian Hero:
\begin{itemize}
    \item \textbf{Company Name}: Nama perusahaan yang akan muncul di footer, tab browser, dan elemen lain secara konsisten.
    \item \textbf{Company Logo}: Upload logo baru untuk mengganti logo di header, footer, dan ikon browser (favicon). Format yang didukung: PNG, JPG, atau SVG.
    \item \textbf{Hero Title/Subtitle}: Mengganti teks judul utama dan subjudul default pada banner utama website.
    \item \textbf{Visi Misi Label/Title}: Mengubah teks label kecil dan judul besar bagian Visi Misi (misal: dari "Visi Misi" menjadi "Filosofi Kami" atau "Misi Perusahaan").
\end{itemize}

\subsection{Halaman 2: Produk, Klien \& Riwayat}
Tab ini mengelola label dan judul untuk bagian produk, klien, dan riwayat perusahaan:
\begin{itemize}
    \item \textbf{Produk Label/Title}: Mengubah teks label kecil dan judul besar untuk bagian Produk.
    \item \textbf{Klien Label/Title}: Mengubah teks label kecil dan judul besar untuk bagian Our Client.
    \item \textbf{Riwayat Label/Title}: Mengubah teks label kecil dan judul besar untuk bagian Company History.
\end{itemize}

\subsection{Halaman 3: Testimoni \& Kontak}
Tab ini mengelola label dan judul untuk bagian testimoni dan kontak:
\begin{itemize}
    \item \textbf{Testimoni Label/Title}: Mengubah teks label kecil dan judul besar untuk bagian Testimoni.
    \item \textbf{Kontak Label/Title}: Mengubah teks label kecil dan judul besar untuk bagian Kontak.
\end{itemize}

% Placeholder untuk screenshot halaman pengaturan situs (tab 1)
\begin{figure}[H]
    \centering
    \includegraphics[width=0.8\textwidth]{placeholder_site_settings_tab1.png}
    \caption{Halaman Pengaturan Situs - Tab 1: Info Dasar \& Hero}
    \label{fig:site-settings-tab1}
\end{figure}

% Placeholder untuk screenshot halaman pengaturan situs (tab 2)
\begin{figure}[H]
    \centering
    \includegraphics[width=0.8\textwidth]{placeholder_site_settings_tab2.png}
    \caption{Halaman Pengaturan Situs - Tab 2: Produk, Klien \& Riwayat}
    \label{fig:site-settings-tab2}
\end{figure}

% Placeholder untuk screenshot halaman pengaturan situs (tab 3)
\begin{figure}[H]
    \centering
    \includegraphics[width=0.8\textwidth]{placeholder_site_settings_tab3.png}
    \caption{Halaman Pengaturan Situs - Tab 3: Testimoni \& Kontak}
    \label{fig:site-settings-tab3}
\end{figure}

\section{Menyimpan Pengaturan}
Tombol navigasi halaman ada di bagian bawah form. Setelah selesai melakukan perubahan di halaman manapun, klik tombol \menu{Simpan Pengaturan} di bagian paling bawah. Perubahan ini bersifat global dan langsung diterapkan ke seluruh halaman website.

\section{Tips Pengaturan Situs}
\begin{itemize}
    \item Backup logo perusahaan sebelum mengganti logo di sistem
    \item Gunakan nama perusahaan yang konsisten di seluruh elemen
    \item Pastikan label dan judul yang digunakan mencerminkan identitas perusahaan
    \item Gunakan logo dengan resolusi tinggi untuk hasil terbaik di semua perangkat
    \item Simpan secara berkala saat melakukan banyak perubahan
    \item Uji perubahan di berbagai perangkat untuk memastikan tampilan optimal
\end{itemize}

%----------------------------------------------------------------------------------------
%   BAB 10: MANAJEMEN ADMIN (SUPER ADMIN)
%----------------------------------------------------------------------------------------
\chapter{Manajemen Admin}

\begin{warningbox}[Akses Terbatas]
Menu ini hanya terlihat dan dapat diakses jika Anda login sebagai \textbf{Super Admin}. Pastikan hanya orang terpercaya yang memiliki akses super admin.
\end{warningbox}

Fitur ini digunakan untuk menambah personil yang boleh mengakses panel admin. Manajemen admin penting untuk mengatur pembagian tugas dalam pengelolaan website.

\section{Menambah Admin}
\begin{enumerate}
    \item Masuk ke menu \menu{Manajemen Admin}.
    \item Klik tombol \menu{+ Tambah Admin} di pojok kanan atas.
    \item Isi formulir yang muncul:
        \begin{itemize}
            \item \textbf{Username}: Nama pengguna unik yang akan digunakan untuk login (hanya huruf, angka, dan underscore yang diperbolehkan).
            \item \textbf{Password}: Kata sandi (minimal 6 karakter). Pastikan menggunakan password yang kuat dan aman.
        \end{itemize}
    \item Klik tombol \menu{Tambah Admin} untuk menyimpan akun admin baru.
    \item Setelah akun dibuat, informasikan username dan password kepada admin yang bersangkutan secara aman.
\end{enumerate}

% Placeholder untuk screenshot form menambah admin
\begin{figure}[H]
    \centering
    \includegraphics[width=0.8\textwidth]{placeholder_add_admin.png}
    \caption{Form Menambah Akun Admin Baru}
    \label{fig:add-admin}
\end{figure}

\section{Mengedit Admin}
\begin{enumerate}
    \item Klik ikon \textbf{Pensil} pada kartu admin yang ingin diedit.
    \item Anda dapat mengubah username jika diperlukan.
    \item \textbf{Password}: Kosongkan field password jika tidak ingin mengubah password. Isi field password hanya jika ingin mereset atau mengganti password pengguna tersebut.
    \item Klik tombol \menu{Simpan Perubahan} untuk menyimpan update informasi admin.
\end{enumerate}

% Placeholder untuk screenshot form mengedit admin
\begin{figure}[H]
    \centering
    \includegraphics[width=0.8\textwidth]{placeholder_edit_admin.png}
    \caption{Form Mengedit Akun Admin}
    \label{fig:edit-admin}
\end{figure}

\section{Menghapus Admin}
\begin{enumerate}
    \item Klik ikon \textbf{Sampah} pada kartu admin yang ingin dihapus.
    \item Konfirmasi penghapusan pada dialog yang muncul.
    \item Akun admin tersebut tidak akan bisa login lagi ke sistem setelah dihapus.
    \item Pastikan untuk mengonfirmasi bahwa admin yang dihapus tidak lagi membutuhkan akses ke sistem sebelum melanjutkan.
\end{enumerate}

% Placeholder untuk screenshot daftar admin dan opsi hapus
\begin{figure}[H]
    \centering
    \includegraphics[width=0.8\textwidth]{placeholder_delete_admin.png}
    \caption{Daftar Admin dan Opsi Hapus}
    \label{fig:delete-admin}
\end{figure}

\section{Tips Manajemen Admin}
\begin{itemize}
    \item Hanya berikan akses admin kepada orang yang benar-benar dipercaya
    \item Gunakan username yang mudah dikenali dan menggambarkan peran pengguna
    \item Ganti password admin secara berkala untuk keamanan
    \item Jangan pernah membagikan akun admin melalui saluran komunikasi yang tidak aman
    \item Hapus akun admin yang sudah tidak aktif atau tidak lagi membutuhkan akses
    \item Gunakan kombinasi password yang kuat (gabungan huruf besar, huruf kecil, angka, dan simbol)
    \item Pertimbangkan untuk membuat kebijakan penggantian password secara berkala
\end{itemize}

%----------------------------------------------------------------------------------------
%   BAB 11: PENUTUP
%----------------------------------------------------------------------------------------
\chapter{Penutup}

Demikian panduan lengkap penggunaan sistem informasi profil perusahaan PT. Surya Kencana Gemilang Teknik. Dokumen ini dirancang sebagai referensi komprehensif untuk penggunaan dan pengelolaan sistem secara efektif dan efisien.

Sistem ini dibangun dengan teknologi modern untuk memastikan fleksibilitas, kinerja optimal, serta kemudahan dalam pengelolaan konten perusahaan. Dengan berbagai fitur yang telah disediakan, tim manajemen dapat secara mandiri memperbarui informasi perusahaan kepada publik tanpa ketergantungan pada pihak eksternal.

\section{Dukungan Teknis}
Untuk bantuan teknis lebih lanjut, permintaan fitur tambahan, atau pelaporan kendala (bug), silakan menghubungi tim pengembang yang tertera pada halaman depan dokumen ini. Kami siap memberikan dukungan yang diperlukan untuk memastikan sistem berfungsi optimal sesuai kebutuhan perusahaan.

\section{Pembaruan Dokumentasi}
Dokumentasi ini akan terus diperbarui seiring dengan pengembangan sistem. Pastikan untuk selalu menggunakan versi terbaru dari dokumen ini sebagai referensi utama dalam penggunaan sistem. Setiap pembaruan akan mencakup penjelasan tentang fitur-fitur baru, perubahan signifikan, dan panduan penggunaan yang ditingkatkan.

\section{Kesimpulan}
Dengan mengikuti panduan dalam dokumen ini, pengguna diharapkan dapat memanfaatkan seluruh fitur sistem secara maksimal untuk mendukung kegiatan pengelolaan informasi perusahaan. Keberhasilan penggunaan sistem sangat bergantung pada pemahaman dan implementasi yang tepat terhadap panduan yang telah disediakan.

\end{document}